% Opcje klasy 'iithesis' opisane sa w komentarzach w pliku klasy. Za ich pomoca
% ustawia sie przede wszystkim jezyk oraz rodzaj (lic/inz/mgr) pracy.
\documentclass[shortabstract,inz]{iithesis}
\usepackage[utf8]{inputenc}

%%%%% DANE DO STRONY TYTUŁOWEJ
% Niezaleznie od jezyka pracy wybranego w opcjach klasy, tytul i streszczenie
% pracy nalezy podac zarowno w jezyku polskim, jak i angielskim.
% Pamietaj o madrym (zgodnym z logicznym rozbiorem zdania oraz estetyka) recznym
% zlamaniu wierszy w temacie pracy, zwlaszcza tego w jezyku pracy. Uzyj do tego
% polecenia \fmlinebreak.
\polishtitle    {Implementacja aplikacji internetowej do organizacji wizyt duszpasterskich w parafiach}
\englishtitle   {Implementation of web app for organising pastoral visits in parishes}
\polishabstract {\ldots}
\englishabstract{\ldots}
% w pracach wielu autorow nazwiska mozna oddzielic poleceniem \and
\author         {Michał Chawar}
% w przypadku kilku promotorow, lub koniecznosci podania ich afiliacji, linie
% w ponizszym poleceniu mozna zlamac poleceniem \fmlinebreak
\advisor        {dr Wiktor Zychla}
%\date          {}                     % Data zlozenia pracy
% Dane do oswiadczenia o autorskim wykonaniu
\transcriptnum {337368}                     % Numer indeksu
\advisorgen    {dr. Wiktora Zychli} % Nazwisko promotora w dopelniaczu
%%%%%

%%%%% WLASNE DODATKOWE PAKIETY
%
%\usepackage{graphicx,listings,amsmath,amssymb,amsthm,amsfonts,tikz}
%
%%%%% WŁASNE DEFINICJE I POLECENIA
%
%\theoremstyle{definition} \newtheorem{definition}{Definition}[chapter]
%\theoremstyle{remark} \newtheorem{remark}[definition]{Observation}
%\theoremstyle{plain} \newtheorem{theorem}[definition]{Theorem}
%\theoremstyle{plain} \newtheorem{lemma}[definition]{Lemma}
%\renewcommand \qedsymbol {\ensuremath{\square}}
% ...
%%%%%

\begin{document}

%%%%% POCZĄTEK ZASADNICZEGO TEKSTU PRACY

\chapter{Wprowadzenie}

\section{Przedstawienie problematyki}
\section{Plan pracy} 

\chapter{Opis i analiza zagadnienia}

\section{Historyjki użytkownika}
\section{Dodatkowe wymagania funkcjonalne}
\section{Wymagania niefunkcjonalne }

\chapter{Porównanie z innymi implementacjami} 
(może być również częścią wprowadzenia przed planem pracy)

\chapter{Implementacja}

\section{Spis użytych technologii i narzędzi (narzędzia pracy grupowej, github, narzędzia do przygotowania makiet itd.)}
\section{Model danych (+ diagramy)}
\section{Architektura (+ diagramy)}
\section{Opis rozwiazania wybranych problemów }

\chapter{Instrukcja użytkownika (+ zrzuty ekranów)}
\chapter{Część dla programisty}

\section{Instrukcja instalacji}
\section{Statystyki, testy jednostkowe }

\chapter{Podsumowanie i dalszy rozwój} 

%%%%% BIBLIOGRAFIA

%\begin{thebibliography}{1}
%\bibitem{example} \ldots
%\end{thebibliography}

\end{document}