% Opcje klasy 'iithesis' opisane sa w komentarzach w pliku klasy. Za ich pomoca
% ustawia sie przede wszystkim jezyk oraz rodzaj (lic/inz/mgr) pracy.
\documentclass[shortabstract,inz]{iithesis}
\usepackage[utf8]{inputenc}

%%%%% DANE DO STRONY TYTUŁOWEJ
% Niezaleznie od jezyka pracy wybranego w opcjach klasy, tytul i streszczenie
% pracy nalezy podac zarowno w jezyku polskim, jak i angielskim.
% Pamietaj o madrym (zgodnym z logicznym rozbiorem zdania oraz estetyka) recznym
% zlamaniu wierszy w temacie pracy, zwlaszcza tego w jezyku pracy. Uzyj do tego
% polecenia \fmlinebreak.
\polishtitle{Implementacja aplikacji internetowej do organizacji wizyt duszpasterskich w parafiach}
\englishtitle{Implementation of web app for organising pastoral visits in parishes}
\polishabstract{\ldots}
\englishabstract{\ldots}
% w pracach wielu autorow nazwiska mozna oddzielic poleceniem \and
\author{Michał Chawar}
% w przypadku kilku promotorow, lub koniecznosci podania ich afiliacji, linie
% w ponizszym poleceniu mozna zlamac poleceniem \fmlinebreak
\advisor{dr Wiktor Zychla}
%\date          {}                     % Data zlozenia pracy
% Dane do oswiadczenia o autorskim wykonaniu
\transcriptnum{337368}                     % Numer indeksu
\advisorgen{dr. Wiktora Zychli} % Nazwisko promotora w dopelniaczu
%%%%%

%%%%% WLASNE DODATKOWE PAKIETY
%
% \usepackage{graphicx,listings,amsmath,amssymb,amsthm,amsfonts,tikz}
\usepackage{todonotes}
\usepackage{subcaption}
\usepackage{wrapfig}
%
%%%%% WŁASNE DEFINICJE I POLECENIA
%
%\theoremstyle{definition} \newtheorem{definition}{Definition}[chapter]
%\theoremstyle{remark} \newtheorem{remark}[definition]{Observation}
%\theoremstyle{plain} \newtheorem{theorem}[definition]{Theorem}
%\theoremstyle{plain} \newtheorem{lemma}[definition]{Lemma}
%\renewcommand \qedsymbol {\ensuremath{\square}}
% ...
%%%%%

\begin{document}

%%%%% POCZĄTEK ZASADNICZEGO TEKSTU PRACY

\chapter{Wprowadzenie}

\section{Problematyka}

Doroczna wizyta duszpasterska (tzw. kolęda) jest tradycyjnym elementem życia
religijnego w Polsce. Polega ona na odwiedzinach księdza w domach parafian,
podczas których udziela on błogosławieństwa, rozmawia z mieszkańcami oraz
zbiera ofiary na potrzeby parafii. Organizacja wizyt duszpasterskich może być
jednak wyzwaniem logistycznym, zwłaszcza w większych parafiach. Często wymaga
to koordynacji wielu osób, ustalania terminów oraz zarządzania informacjami o
parafianach.

W dzisiejszych czasach w zależności od parafii wizyty duszpasterskie
przeprowadzane są w dwóch modelach:

\begin{itemize}
      \item \textbf{Kolęda tradycyjna}, podczas której duszpasterze starają się
            dotrzeć do wszystkich parafian.
      \item \textbf{Kolęda na zaproszenie}, podczas której duszpasterze odwiedzają
            tylko tych parafian, którzy wcześniej samodzielnie zgłosili chęć
            przyjęcia wizyty.
\end{itemize}

W niniejszej pracy skupiono się na opracowaniu aplikacji internetowej,
wspierającej organizację wizyt duszpasterskich, przeprowadzanych w modelu
\emph{kolędy na zaproszenie}. Gdy dalej mowa o kolędzie, chodzi właśnie o ten
model wizyty duszpasterskiej.

\subsection{Kolęda tradycyjna}
Model tradycyjny stosowany jest w większości parafii w Polsce. Funkcjonuje od
dziesiątek lat i jest dobrze znany zarówno duszpasterzom, jak i parafianom. W
tym modelu księża starają się odwiedzić wszystkich parafian w określonym
czasie, zwyczajowo w okresie Bożego Narodzenia. Czynią to według ustalonego
harmonogramu, zazwyczaj prostego i niewymagającego skomplikowanej logistyki.

Podczas gdy to podejście ma wiele zalet duszpasterskich, to jednak jest mocno
czasochłonne i obciążające dla duszpasterzy oraz wiernych. Cierpi na nim często
też jakoś wizyt, zarówno z powodu bardziej ścisłego ograniczenia czasowego na
wizytę, jak i znacznego odsetka parafian, którzy wizytę przyjmują z obowiązku
lub przyzwyczajenia, a nie z potrzeby duchowej. W większych parafiach taka
forma dodatkowo angażuje ministrantów do dodatkowej pracy organizacyjnej —
zapowiadania kolędy, czyli uprzedniego chodzenia od drzwi do drzwi i
informowania o terminie wizyty oraz zbierania wstępnych deklaracji chęci
przyjęcia wizyty. Jest to również główny sposób wczesnego prognozowania liczby
wizyt danego dnia.

\subsection{Kolęda na zaproszenie}
Model ten zyskuje na popularności w ostatnich latach, zwłaszcza w większych
miastach, gdzie parafianie mogą mieć bardziej zróżnicowane potrzeby, a odsetek
odwiedzanych domów może być mniejszy. Znany jest również poza granicami Polski.
W tej formie parafianie sami zgłaszają chęć przyjęcia wizyty duszpasterskiej
poprzez odpowiedni formularz (papierowy lub elektroniczny).

Z perspektywy organizacyjnej, model ten jest znacznie bardziej efektywny.
Pozwala dokładnie planować wizyty na każdy dzień, co zmniejsza obciążenie
duszpasterzy, zwiększając jednocześnie kontrolę nad rozłożeniem wizyt w czasie
oraz ich równomierność. Likwiduje dodatkowo potrzebę uprzedniego zapowiadania
kolędy przez ministrantów, tworząc jednak nową odpowiedzialność za szczegółowe
zaplanowanie porządku na dany dzień. Tym zajmuje się albo sam duszpasterz, albo
wyznaczona do tego osoba (np. kościelny lub inny pracownik parafii) jako
koordynator wizyty duszpasterskiej.

Kolęda na zaproszenie generuje jednak całkiem nową potrzebę — skutecznego
zarządzania zgłoszeniami parafian. W większych parafiach liczba zgłoszeń może
być znaczna, co wymaga odpowiednich narzędzi do ich rejestracji, przetwarzania
i planowania wizyt. Wymaga to również odpowiedniej komunikacji z parafianami,
aby zapewnić im jasne informacje o terminach wizyt oraz ewentualnych zmianach w
harmonogramie. Dokładnie tym wymaganiom ma na celu sprostać opisywana w
niniejszej pracy aplikacja internetowa.

\section{Plan pracy}

\chapter{Opis i analiza zagadnienia}
\label{sec:rozdzial_II}

Zagadnienie organizacji wizyty duszpasterskiej w modelu \emph{kolędy na
      zaproszenie} jest dosyć skomplikowane i wymaga szeregu funkcjonujących i ściśle
ze sobą współpracujących narzędzi (przede wszystkim do zbierania zgłoszeń,
układania planów i informowania o nich). Dodatkowo potrzeba również personelu,
który zajmie się zarówno wstępnym zaplanowaniem kolędy i ustaleniem
harmonogramu, jak i późniejszą koordynacją przeprowadzania wizyty według planu.

Brak któregokolwiek z tych elementów znacznie utrudnia, a często nawet
uniemożliwia, organizację kolędy na zaproszenie w sposób efektywny i
zadowalający. Z kolei niewłaściwy dobór tych narzędzi lub nieścisła współpraca
między nimi może prowadzić do licznych problemów organizacyjnych, błędów w
planowaniu i komunikacji z parafianami.

\todo{Jakiś ładny diagramik z potencjalnymi problemami?}

Aby model kolędy na zaproszenie osiągał rzeczywistą przewagę nad modelem
tradycyjnym, wszystkie te elementy muszą być starannie dobrane i nadzorowane. W
wielu parafiach brakuje jednak odpowiednich narzędzi i zasobów. Nie ma też na
rynku dedykowanych rozwiązań informatycznych, które kompleksowo wspierałyby ten
model wizyty duszpasterskiej. Z tego powodu często korzysta się z rozwiązań
prymitywnych, prowizorycznych lub niedostosowanych do specyficznych potrzeb
kolędy na zaproszenie, co z kolei nie tylko prowadzi do licznych trudności
organizacyjnych, ale dodatkowo powoduje frustrację zarówno wśród parafian, jak
i koordynatorów kolędy.

Opisywana aplikacja wychodzi naprzeciw tym wyzwaniom, oferując kompleksowe
narzędzie do zarządzania kolędą, które integruje wszystkie niezbędne funkcje w
jednym miejscu. Dzięki temu parafie mogą skutecznie organizować wizyty
duszpasterskie, minimalizując ryzyko błędów i usprawniając komunikację z
parafianami.

\section{Historyjki użytkownika}

\section{Wymagania ogólne}

Jak wspomniano wcześniej, potrzebne narzędzia do organizacji kolędy muszą być
ze sobą ściśle zintegrowane i współpracujące. Brak takich na rynku, powiązany z
koniecznością przeprowadzenia kolędy, był główną motywacją do stworzenia i
rozwoju opisywanej aplikacji. Już w początkowym etapie projektowania systemu
wyklarowały się najważniejsze wymagania, które musiał spełniać, aby nie tylko
skutecznie pomagać w organizacji kolędy, ale również mieć realną szansę na
szerokie zastosowanie w różnych parafiach.

\subsection{Jedność}

Aplikacja musi integrować wszystkie kluczowe funkcje potrzebne do organizacji
kolędy na zaproszenie w jednym miejscu. Oznacza to, że minimalnie powinna
implementować funkcjonalności:

\begin{itemize}
      \item zbierania zgłoszeń od parafian,
      \item zarządzania zebranymi zgłoszeniami oraz wprowadzania nowych,
      \item tworzenia i edycji harmonogramu.
\end{itemize}

Niemniej jednak biorąc pod uwagę współczesne możliwości technologiczne oraz
dążenie do maksymalnej automatyzacji i uproszczenia procesu organizacji możemy
na podstawie implementacji tych trzech punktów zbudować nowoczesny system,
który dodatkowo będzie oferował wiele innych przydatnych funkcji:

\begin{itemize}
      \item zautomatyzowane i zindywidualizowane powiadamianie mailowe,
      \item portal dla parafian, umożliwiający wgląd w dane swojego zgłoszenia, jego
            status, termin wizyty, a nawet przewidywane godziny,
      \item przeprowadzanie wizyty według planu,
      \item zarządzanie personelem zaangażowanym w kolędę,
      \item raportowanie i statystyki dotyczące przebiegu kolędy.
\end{itemize}

Powyższy zestaw funkcji zebrany w jednym miejscu pozwala na kompleksowe
zarządzanie kolędą oraz zapewnia znaczną przewagę nad rozbitymi,
prowizorycznymi narzędziami, które często są wykorzystywane w parafiach.

\subsection{Intuicyjność}

Aplikacja powinna być prosta i intuicyjna w obsłudze, tak aby nawet osoby
nieposiadające zaawansowanych umiejętności technicznych mogły z niej skutecznie
korzystać. Interfejs użytkownika musi być przejrzysty, zrozumiały i jednolity.

Należy jednak dołożyć wszelkich starań, aby nie upraszczać zbytnio
funkcjonalności aplikacji kosztem jej użyteczności. Wdrożenie zbyt
ograniczonego zestawu funkcji może sprawić, że aplikacja nie będzie w stanie
spełnić wszystkie potrzeby parafii, co z kolei może prowadzić do jej odrzucenia
na rzecz bardziej rozbudowanych, choć mniej zintegrowanych rozwiązań.

\subsection{Poprawność}

Aplikacja musi działać niezawodnie i bezbłędnie, zapewniając poprawne
funkcjonowanie wszystkich swoich funkcji. Wszelkie błędy lub awarie mogą
prowadzić do poważnych problemów organizacyjnych, a nawet do utraty zaufania
parafian. Dlatego tak ważne jest, aby aplikacja była starannie przetestowana i
regularnie aktualizowana, aby zapewnić jej stabilność i niezawodność.

Oprócz samej poprawności implementacji aplikacji należy wziąć pod uwagę, że
dane wprowadzone do systemu muszą być również poprawne i spójne. W tym celu
aplikacja powinna implementować mechanizmy walidacji danych, które zapewnią, że
wprowadzone informacje są zgodne z określonymi standardami i nie zawierają
błędów.

Ostatecznie aplikacja będzie wykorzystywana przez koordynatorów kolędy, którzy
mogą nie mieć zaawansowanych umiejętności technicznych. Z tego powodu należy
przyłożyć szczególną uwagę do upewnienia się, że operacje dostępne z poziomu
interfejsu użytkownika są dobrze przemyślane i nie prowadzą do niezamierzonych
konsekwencji, a także dobrze opisane, aby użytkownicy mogli z nich korzystać
bez ryzyka popełnienia błędów.

\subsection{Reużywalność}

Aplikacja powinna być zaprojektowana w sposób umożliwiający jej łatwe
dostosowanie i ponowne wykorzystanie w różnych parafiach. Oznacza to, że
powinna być elastyczna i konfigurowalna, a także umożliwiać coroczne
przeprowadzanie wizyt kolędowych bez konieczności manualnej rekonfiguracji lub
ponownego wdrażania. Dzięki temu parafie będą mogły korzystać z aplikacji przez
wiele lat, co znacznie zwiększy jej wartość i użyteczność.

Sam fakt wykorzystania w różnych parafiach wymusza również konieczność
uwzględnienia tej możliwości w systemie, aby różne parafie mogły korzystać z
aplikacji jednocześnie, bez konieczności tworzenia odrębnych instancji lub
wersji aplikacji dla każdej z nich.

\section{Wymagania funkcjonalne}

Podsumowując powyższe rozważania, aplikacja powinna spełniać następujące
wymagania funkcjonalne:

\begin{itemize}
      \item umożliwiać parafianom zgłaszanie chęci uczestnictwa w kolędzie na zaproszenie
            poprzez formularz online,
      \item pozwalać koordynatorom kolędy na przeglądanie, edytowanie i zarządzanie
            zebranymi zgłoszeniami,
      \item umożliwiać tworzenie i edycję harmonogramu wizyt duszpasterskich na podstawie
            zebranych zgłoszeń,
      \item automatycznie wysyłać powiadomienia mailowe do parafian o potwierdzeniu
            przyjęcia zgłoszenia,
      \item umożliwiać manualne wysyłanie powiadomień mailowych z informacjami o zmianie
            danych zgłoszenia, zaplanowaniu zgłoszenia, itp.,
      \item oferować portal dla parafian, gdzie mogą oni sprawdzać status swojego
            zgłoszenia, termin wizyty oraz przewidywane godziny,
      \item wspierać koordynację personelu zaangażowanego w kolędę, umożliwiając
            przypisywanie zadań i monitorowanie postępów,
      \item generować raporty i statystyki dotyczące przebiegu kolędy, takie jak liczba
            zgłoszeń, liczba przeprowadzonych wizyt itp.,
      \item umożliwiać konfigurację i dostosowanie aplikacji do specyficznych potrzeb
            różnych parafii,
      \item zapewniać możliwość corocznego przeprowadzania kolędy bez konieczności
            manualnej rekonfiguracji lub ponownego wdrażania aplikacji,
      \item umożliwiać jednoczesne korzystanie z aplikacji przez różne parafie, bez
            konieczności tworzenia odrębnych instancji lub wersji aplikacji dla każdej z
            nich.
\end{itemize}

\section{Wymagania niefunkcjonalne }

Oprócz wymagań funkcjonalnych, aplikacja powinna również spełniać następujące
wymagania niefunkcjonalne:

\begin{itemize}
      \item być dostępna online, aby parafie i parafianie mogli z niej korzystać z
            dowolnego miejsca i o dowolnym czasie,
      \item być skalowalna, aby mogła obsługiwać rosnącą liczbę użytkowników i zgłoszeń bez
            utraty wydajności,
      \item być bezpieczna, aby chronić dane osobowe parafian i zapewnić poufność
            informacji,
      \item być zgodna z obowiązującymi przepisami dotyczącymi ochrony danych osobowych,
            takimi jak RODO,
      \item być łatwa w utrzymaniu i aktualizacji, aby zapewnić jej długotrwałe
            funkcjonowanie i możliwość dostosowywania do zmieniających się potrzeb parafii.
\end{itemize}

\chapter{Implementacja}
\label{sec:rozdzial_III}

\section{Technologie i narzędzia}

Aplikacja została zaimplementowana w technologii ASP.NET Core MVC w środowisku
.NET 8. Do zarządzania bazą danych wykorzystano Microsoft SQL Server 2022.
Interfejs użytkownika został zbudowany przy użyciu TailwindCSS, zapewniającego
nowoczesny i responsywny wygląd, oraz Vue.js do obsługi interaktywnych
komponentów i wieloetapowych procedur. Całość projektu została objęta
konteneryzacją przy użyciu narzędzia Docker, co umożliwia łatwe wdrożenie i
utrzymanie aplikacji.

\subsection{Strona kliencka}
\label{sec:frontend_opis}

Część frontendowa aplikacji wykonana jest w dwóch metodykach. Pierwsza z nich
przeznaczona jest do tworzenia statycznych stron HTML, które są renderowane po
stronie serwera w sposób typowy dla wzorca MVC\@. Drugie podejście łączy
wstępną generację HTML z dynamicznym uzupełnianiem treści po stronie klienta
przy użyciu biblioteki Vue.js. Takie podejście zostało zastosowane w miejscach,
gdzie wymagana jest większa interaktywność, na przykład w formularzach
wieloetapowych czy dynamicznych listach.

\subsubsection{Razor}

Do tworzenia statycznych stron HTML wykorzystano silnik szablonów Razor, który
jest integralną częścią frameworka ASP.NET Core MVC\@. Razor umożliwia łączenie
kodu C\# z HTML w sposób czytelny i efektywny, co pozwala na dynamiczne
generowanie treści stron na serwerze przed ich wysłaniem do przeglądarki
klienta.

Z perspektywy klienta, strony wygenerowane za pomocą Razor są tradycyjnymi
stronami HTML, które mogą zawierać osadzone skrypty JavaScript i style CSS\@.
Dzięki temu możliwe jest tworzenie responsywnych i interaktywnych interfejsów
użytkownika, nawet jeśli główna logika renderowania stron odbywa się po stronie
serwera.

\subsubsection{Vue.js}
\label{sec:vue_opis}

Niektóre części aplikacji wymagają większej interaktywności i dynamicznego
zarządzania stanem interfejsu użytkownika (w celu zachowania wymogu
intuicyjności). Do ich implementacji wykorzystano bibliotekę Vue.js.

Strony te są początkowo generowane na serwerze przy użyciu Razor, a następnie
po załadowaniu w przeglądarce klienta, Vue.js przejmuje kontrolę nad
interaktywnymi elementami interfejsu użytkownika. Dzięki temu możliwe jest
dynamiczne aktualizowanie treści, obsługa zdarzeń użytkownika oraz zarządzanie
stanem aplikacji bez konieczności ponownego ładowania całej strony.

Aplikacje Vue.js są implementowane bezpośrednio w tagach \textit{script} w
niektórych plikach widoków. Ładowany jest wówczas globalny plik biblioteki,
który udostępnia wszelkie funkcjonalności jako właściwości globalnego obiektu
Vue. Następnie za jego pomocą tworzone są instancje aplikacji Vue, które są
przypisywane do określonych elementów DOM na stronie. Ten proces następuje w
przeglądarce użytkownika, co pozwala na płynne przejście od statycznego
renderowania do dynamicznej interaktywności.

W opisywanym projekcie aplikacje Vue.js używane są na dwa sposoby:

\begin{itemize}
      \item do zwiększania interaktywności prostych bądź zaawansowanych elementów
            interfejsu po stronie klienta bez potrzeby komunikacji z serwerem,
      \item do zarządzania bardziej złożonymi komponentami interfejsu, które wymagają
            komunikacji z serwerem w celu pobierania lub wysyłania danych.
\end{itemize}

W pierwszym przypadku często zachowywana jest logika renderowania po stronie
serwera, włącznie z tworzeniem formularzy za pomocą pomocników HTML ASP.NET
Core przy użyciu modeli widoków. Vue.js jest wtedy wykorzystywany do obsługi
interakcji użytkownika, takich jak walidacja danych wprowadzanych w
formularzach, dynamiczne dodawanie lub usuwanie elementów listy, czy
aktualizacja widoku na podstawie działań użytkownika bez konieczności ponownego
ładowania strony. Czasem jednak dane osadzane są bezpośrednio w zmiennej
JavaScript w widoku (po przekonwertowaniu do JSON), co pozwala na pełną
kontrolę nad renderowaniem interfejsu po stronie klienta przez samą aplikację
Vue.

W drugim przypadku Vue.js zarządza bardziej złożonymi komponentami, które
wymagają komunikacji z serwerem. Wówczas aplikacja przypomina bardziej wzorzec
SPA (Single Page Application), gdzie Vue.js odpowiada za renderowanie
interfejsu użytkownika, a komunikacja z serwerem odbywa się za pomocą
asynchronicznych żądań HTTP (przy użyciu Fetch API). Dane są pobierane z
serwera w formacie JSON, a następnie wykorzystywane do aktualizacji widoku w
czasie rzeczywistym. Podobnie, dane wprowadzone przez użytkownika są wysyłane z
powrotem na serwer w formacie JSON, gdzie są przetwarzane i zapisywane w bazie
danych. Nie jest to jednak pełna aplikacja SPA, ponieważ nawigacja między
różnymi stronami nadal odbywa się poprzez tradycyjne przeładowanie strony.

\subsubsection{JQuery i JavaScript}

Do obsługi prostych interakcji na stronach wykorzystano również bibliotekę
jQuery oraz czysty JavaScript. W miejscach, gdzie nie jest wymagana pełna
funkcjonalność Vue.js, jQuery pozwala na szybkie i efektywne manipulowanie
elementami DOM oraz obsługę zdarzeń. Dodatkowo odpowiada za walidację
formularzy po stronie klienta, co poprawia doświadczenie użytkownika poprzez
natychmiastowe informowanie o błędach przed wysłaniem danych na serwer.

JavaScript jest również używany do implementacji powszechnych funkcji dla
aplikacji, znajdujących się w plikach zewnętrznych, które są dołączane do
odpowiednich widoków.

\subsubsection{TailwindCSS}

Do stylizacji interfejsu użytkownika wykorzystano framework CSS o nazwie
TailwindCSS\@. Jest to narzędzie oparte na podejściu utility-first, które
umożliwia szybkie tworzenie responsywnych i estetycznych interfejsów
użytkownika poprzez stosowanie gotowych klas CSS bez konieczności pisania
własnych stylów od podstaw.

Głównymi zaletami TailwindCSS są jego elastyczność i możliwość tworzenia
responsywnych projektów. Framework oferuje szeroki zestaw klas, które pozwalają
na precyzyjne kontrolowanie wyglądu elementów interfejsu, takich jak marginesy,
wypełnienia, kolory, typografia i układ. Dzięki temu aplikacja została w prosty
sposób dostosowana do różnych rozmiarów ekranów, zapewniając optymalne
doświadczenie użytkownika na urządzeniach mobilnych, tabletach i komputerach
stacjonarnych.

Nad to użyto również wtyczki DaisyUI, która rozszerza możliwości TailwindCSS o
gotowe komponenty UI, takie jak przyciski, formularze, karty i nawigacje,
tworzone za pomocą określonych klas. Dzięki temu proces tworzenia interfejsu
użytkownika był szybszy i bardziej efektywny, pozwalając skupić się na
funkcjonalności aplikacji zamiast na szczegółach stylizacji.

\subsection{Strona serwerowa}

Część backendowa aplikacji została zaimplementowana przy użyciu frameworka
ASP.NET Core, opartego na platformie \.NET w wersji 8. Wykorzystano w nim różne
biblioteki i narzędzia dostępne w ekosystemie \.NET, aby zapewnić wydajność,
skalowalność i bezpieczeństwo aplikacji.

\subsubsection{ASP.NET Core}

Framework ASP.NET Core umożliwia tworzenie aplikacji webowych zgodnych z
wzorcem Model-View-Controller (MVC), co pozwala na oddzielenie logiki
biznesowej od warstwy prezentacji i danych. Nadal pozwala przy tym udostępniać
interfejs API w obrębie tej samej aplikacji, co zostało wykorzystane w
projekcie. ASP.NET Core oferuje wbudowane mechanizmy do obsługi routingu,
autoryzacji, uwierzytelniania oraz zarządzania sesjami, co ułatwia tworzenie
bezpiecznych i wydajnych aplikacji webowych. Dodatkowo, framework ten jest
wysoce konfigurowalny i wspiera nowoczesne praktyki programiste, takie jak
wstrzykiwanie zależności i middleware.

\paragraph{Entity Framework Core}

Do komunikacji z bazą danych wykorzystano Entity Framework Core (EF Core),
który jest popularnym narzędziem ORM (Object-Relational Mapping) dla platformy
\.NET\@. EF Core umożliwia programistom pracę z bazą danych za pomocą obiektów
C\#, eliminując potrzebę pisania bezpośrednich zapytań SQL\@. Dzięki temu
proces tworzenia, odczytu, aktualizacji i usuwania danych (CRUD) staje się
bardziej intuicyjny i zintegrowany z logiką aplikacji.

\paragraph{ASP.NET Core Identity}

Do zarządzania uwierzytelnianiem i autoryzacją użytkowników wykorzystano
bibliotekę ASP.NET Core Identity. Jest to kompleksowe rozwiązanie, które
umożliwia tworzenie i zarządzanie kontami użytkowników, obsługę ról oraz
implementację mechanizmów bezpieczeństwa, takich jak resetowanie haseł czy
weryfikacja dwuetapowa. ASP.NET Core Identity integruje się bezproblemowo z
frameworkiem ASP.NET Core, co pozwala na łatwe dodanie funkcji logowania i
zarządzania użytkownikami do aplikacji webowej oraz przechowywanie ich danych w
bazie danych za pośrednictwem Entity Framework Core.

\paragraph{WebOptimizer}

Do optymalizacji dostarczania statycznych plików JavaScript zastosowano
bibliotekę WebOptimizer. Narzędzie to umożliwia minifikację, łączenie i
kompresję plików statycznych, co prowadzi do zmniejszenia rozmiaru przesyłanych
zasobów i przyspieszenia ładowania stron internetowych. WebOptimizer
automatycznie przetwarza pliki podczas uruchamiania aplikacji, co ułatwia
zarządzanie zasobami i poprawia wydajność aplikacji webowej.

\paragraph{MailKit}

Do obsługi wysyłania wiadomości e-mail z aplikacji wykorzystano bibliotekę
MailKit. Jest to nowoczesne i wydajne narzędzie do obsługi protokołów SMTP,
POP3 i IMAP w środowisku \.NET\@. MailKit oferuje szeroki zakres funkcji,
takich jak tworzenie i wysyłanie wiadomości e-mail, obsługa załączników,
szyfrowanie oraz autoryzacja. Biblioteka ta jest znana ze swojej wydajności i
niezawodności, co czyni ją idealnym wyborem do integracji funkcji e-mail w
aplikacjach webowych.

\paragraph{DataProtection}

Do zapewnienia trwałości kluczy kryptograficznych wykorzystano bibliotekę
ASP.NET Core Data Protection. Dzięki integracji z Entity Framework Core, klucze
są przechowywane w bazie danych, co zapewnia ich persystencję między restartami
aplikacji, umożliwiając zachowanie sesji użytkowników i likwidując wymóg
ponownego logowania po restarcie serwera.

\paragraph{QuestPDF}

Do generowania dokumentów PDF w aplikacji wykorzystano bibliotekę QuestPDF\@.
Jest to nowoczesne narzędzie do tworzenia wysokiej jakości dokumentów PDF w
środowisku \.NET\@. QuestPDF oferuje prosty i intuicyjny interfejs
programistyczny, który umożliwia definiowanie układu i stylu dokumentów za
pomocą kodu C\#. Biblioteka obsługuje różnorodne funkcje, takie jak dodawanie
tekstu, obrazów, tabel i wykresów, co pozwala na tworzenie profesjonalnie
wyglądających raportów i dokumentów bez konieczności korzystania z zewnętrznych
narzędzi do edycji PDF\@.

Narzędzie QuestPDF zostało wykorzystane na licencji Community MIT, która
pozwala na darmowe użycie biblioteki w projektach niekomercyjnych i
komercyjnych, generujących dochód poniżej \$1,000,000 rocznie.

\subsubsection{Microsoft SQL Server 2022}

Aplikacja korzysta z relacyjnej bazy danych Microsoft SQL Server 2022 do
przechowywania wszystkich danych niezbędnych do jej funkcjonowania. Komunikacja
pomiędzy aplikacją a bazą danych odbywa się za pośrednictwem opisanego wyżej
Entity Framework Core, który umożliwia mapowanie obiektów C\# na tabele i
rekordy w bazie danych. Dodatkowo z bazą danych komunikują się narzędzia
ASP.NET Core Identity oraz Data Protection (poprzez integrację z EF Core).

\subsubsection{Traefik}

W środowisku produkcyjnym do zarządzania ruchem sieciowym i obsługi
certyfikatów SSL/TLS wykorzystano narzędzie Traefik\@. Jest to nowoczesny i
wydajny reverse proxy oraz load balancer, który automatycznie wykrywa usługi i
konfiguruje trasowanie ruchu na podstawie reguł zdefiniowanych przez
użytkownika. Traefik zintegrowany jest z Dockerem, co umożliwia dynamiczne
zarządzanie ruchem sieciowym w środowiskach kontenerowych. Dodatkowo, Traefik
oferuje wbudowaną obsługę Let's Encrypt, co pozwala na automatyczne generowanie
i odnawianie certyfikatów SSL/TLS, zapewniając bezpieczną komunikację między
klientami a serwerem.

\subsection{Ciągła integracja i dostarczanie (CI/CD)}
\label{sec:ci_cd_opis_technologii}

Główna część aplikacji została objęta konteneryzacją przy użyciu narzędzia
Docker. Konteneryzacja pozwala na zapakowanie aplikacji wraz ze wszystkimi jej
zależnościami w odizolowane środowisko, co umożliwia łatwe wdrożenie i dalsze
utrzymanie. Dodatkowo za pomocą pliku Dockerfile zdefiniowano proces budowania
obrazu kontenera od podstaw, co zapewnia spójność środowiska uruchomieniowego
aplikacji niezależnie od miejsca jej wdrożenia.

Wraz z kontenerem aplikacji ASP.NET Core przy pomocy narzędzia orkiestracji
kontenerów Docker Compose można w prosty sposób uruchomić również kontener bazy
danych Microsoft SQL Server 2022 oraz (w środowisku produkcyjnym) kontener
Traefik, jednocześnie konfigurując ich współpracę, sieć wewnętrzną oraz
wolumeny do trwałego przechowywania danych.

\subsection{Testy}

\todo{Zrobić testy XD}

\subsection{Zarządzanie kodem źródłowym}

Kod źródłowy aplikacji jest przechowywany w repozytorium Git na platformie
GitHub. Wykorzystano funkcje zarządzania wersjami, takie jak gałęzie i pull
requesty, aby umożliwić współpracę zespołową (w przyszłości) oraz śledzenie
zmian w kodzie. Dodatkowo, repozytorium zawiera dokumentację projektu,
instrukcje dotyczące wdrożenia oraz konfiguracji środowiska deweloperskiego.

\subsection{Narzędzia modelowania diagramów}

Do tworzenia diagramów UML oraz innych wizualizacji na potrzeby tej pracy użyto
dwóch narzędzi:

\begin{itemize}
      \item Structurizr --- do tworzenia diagramów architektury oprogramowania (C4 Model),
      \item Visual Paradigm (w wersji Community) --- do tworzenia pozostałych diagramów
            (np.\ modelu danych).
\end{itemize}

\section{Modele danych}

Poniżej przedstawiono dwa diagramy. Pierwszy odpowiada modelowi dziedzinowemu,
reprezentującemu główne pojęcia oraz ich relacje w kontekście logiki aplikacji.
Drugi odnosi się do modelu obiektowego, który odwzorowuje strukturę encji EF
Core, tym samym determinujących strukturę bazy danych.

\subsection{Model pojęciowy}

Zaprezentowany diagram modelu pojęciowego (rys. \ref{fig:model_pojęciowy})
ilustruje główne pojęcia oraz ich wzajemne relacje w kontekście logiki
aplikacji. Model ten stanowi abstrakcyjną reprezentację struktur danych i
zależności między nimi, niezależnie od konkretnej implementacji technicznej.

Całość modelu mieści się w logicznych granicach jednej \textbf{parafii} —
wymienione pojęcia odnoszą się do zarządzania kolędą w obrębie konkretnej
parafii, a tym samym do jednej domeny aplikacji.

\begin{figure}[h]
      \centering
      \includegraphics[width=\textwidth]{figures/Model pojęciowy.png}
      \caption{Model pojęciowy aplikacji}
      \label{fig:model_pojęciowy}
\end{figure}

\subsubsection{Personel}
Personel parafialny składa się z różnych ról, które mają określone uprawnienia
i obowiązki w kontekście zarządzania kolędą. Główne role to:
\begin{itemize}
      \item \textbf{Administrator} — osoba odpowiedzialna za zarządzanie
            aplikacją, w tym tworzenie i modyfikowanie planów kolędy oraz
            zarządzanie użytkownikami.
      \item \textbf{Ksiądz} — duchowny, który odwiedza parafian podczas kolędy.
            Może zarządzać danym planem i planować wizyty.
      \item \textbf{Ministrant} — osoba, która towarzyszy księdzu podczas wizyt.
            Może być przypisana do konkretnych agend w planie kolędy i na bieżąco
            aktualizować status wizyt.
\end{itemize}

\subsubsection{Plan}
Najbardziej ogólnym pojęciem w zakresie planowania kolędy jest \textbf{Plan}.
Reprezentuje on konkretny plan kolędy dla danego roku. Planów może być wiele,
są zarządzane przez parafialnego administratora aplikacji.

\subsubsection{Harmonogram}
Każdy plan kolędy musi zawierać przynajmniej jeden \textbf{Harmonogram}.
Harmonogramy umożliwiają podział zgłoszeń parafian na różne grupy, które z
kolei mogą służyć do filtrowania zgłoszeń, a także do tworzenia odrębnych
planów wizyt dla różnych księży. Przykładowo można za ich pomocą rozdzielić
kolędę w terminie zasadniczym od kolędy dodatkowej.

\subsubsection{Dzień}
Każdy plan składa się z wielu \textbf{Dni}, które reprezentują konkretne dni w
trakcie trwania kolędy.

\subsubsection{Agenda}
Każdy dzień zawiera wiele \textbf{Agend}, które grupują i porządkują wizyty
danego dnia. Każda agenda jest przypisana do konkretnego dnia i zawiera
informacje o ministrantach towarzyszących podczas wizyt. Agendy są
odpowiednikiem list wizyt, które ksiądz i ministranci realizują w danym dniu
kolędy. W najprostszym rozumieniu mogą służyć więc jako podział wizyt na
księdza pierwszego, drugiego, itd.

\subsubsection{Wizyta}
Podstawowym elementem planu kolędy jest \textbf{Wizyta}. Reprezentuje ona
konkretną wizytę księdza u parafianina. Wizyta może być przypisana do
konkretnej agendy, a tym samym być zaplanowana na konkretny dzień. Kolejność
wizyt w agendzie jest wyznaczana przez liczbę porządkową wizyty. Zasadniczo
przypisana jest też do konkretnego harmonogramu.

\subsubsection{Zgłoszenie, zgłaszający i adres}
Każda wizyta jest powiązana z konkretnym \textbf{Zgłoszeniem} parafianina.
Zgłoszenie zawiera wszystkie niezbędne informacje o parafianinie, takie jak
dane kontaktowe oraz adres zamieszkania (w powiązanych pojęciach). Do
zgłoszenia można dołączyć także dodatkowe uwagi (np. prośby dotyczące wizyty
lub preferencje dotyczące terminu).

\subsection{Model obiektowy}
\label{sec:model_obiektowy}

Poniżej zaprezentowano dwa diagramy modelu obiektowego aplikacji. Pierwszy z
nich (rys.\ \ref{fig:model_obiektowy_centralny}) odnosi się do kontekstu
ogólnego (centralnego), tj. do logiki zarządzania użytkownikami (docelowo w
różnych domenach — parafiach). Drugi (rys.\
\ref{fig:model_obiektowy_parafialny}) przedstawia model obiektowy funkcjonujący
w obrębie konkretnej domeny (kontekst parafialny).

\subsubsection{Kontekst centralny}

\begin{figure}[h]
      \centering
      \includegraphics[width=\textwidth]{figures/Model obiektowy - centralny.png}
      \caption{Model obiektowy aplikacji (kontekst centralny)}
      \label{fig:model_obiektowy_centralny}
\end{figure}

W kontekście centralnym aplikacji zajęto się zarządzaniem użytkownikami oraz
ich rolami w różnych domenach (parafiach). Główne klasy w tym modelu to
\textbf{User}, \textbf{Role} oraz \textbf{ParishEntry}.

\paragraph{User}
Reprezentuje użytkownika aplikacji. Zawiera podstawowe informacje o nim, w
większości dziedziczone z klasy \texttt{IdentityUser} z ASP.NET Core Identity.
Jest przypisany do konkretnej parafii.

Najważniejsze pola klasy \texttt{User} to:
\begin{itemize}
      \item \texttt{UserName} — login użytkownika,
      \item \texttt{Email} — adres e-mail użytkownika,
      \item \texttt{PasswordHash} — hash hasła użytkownika,
      \item \texttt{DisplayName} — nazwa użytkownika.
\end{itemize}

\paragraph{Role}
Reprezentuje role użytkowników w aplikacji. Każda rola definiuje zestaw
uprawnień i obowiązków, które użytkownik posiada w kontekście zarządzania
kolędą.

\texttt{Role} jest enumeracją o następujących wartościach:
\begin{itemize}
      \item \textbf{DefaultUser} — uprawnia do podstawowego dostępu do aplikacji,
      \item \textbf{VisitSupport} — uprawnia do przeprowadzania wizyt
            kolędowych (odpowiednik ministranta),
      \item \textbf{SubmitSupport} — uprawnia do wprowadzania zgłoszeń i zarządzania nimi,
      \item \textbf{Priest} — uprawnia do wszystkich powyższych czynności oraz do
            planowania wizyt,
      \item \textbf{Administrator} — uprawnia do zarządzania aplikacją i wszystkimi
            jej funkcjami.
\end{itemize}

\paragraph{ParishEntry}
Reprezentuje rzeczywistą parafię, która używa aplikacji do zarządzania kolędą.
Do każdej jest przypisany przynajmniej jeden użytkownik.

Najważniejsze pola klasy \texttt{ParishEntry} to:
\begin{itemize}
      \item \texttt{UniqueId} — unikalny identyfikator parafii,
      \item \texttt{ParishName} — nazwa parafii,
      \item \texttt{EncryptedConnectionString} — zaszyfrowany łańcuch połączenia do bazy danych parafii,
      \item \texttt{CreationTime} — data utworzenia parafii w systemie.
\end{itemize}

\subsubsection{Kontekst parafialny}

Kontekst parafialny zbiera wszystkie klasy związane z zarządzaniem kolędą w
obrębie konkretnej parafii. Główne klasy w tym modelu to \textbf{Plan},
\textbf{Schedule}, \textbf{Day}, \textbf{Agenda}, \textbf{Submission} oraz
\textbf{Visit}.

W tym modelu wyróżniają się dwie najistotniejsze grupy encji (w podziale
funkcjonalnym): encje służące do \textit{Przyjmowania zgłoszeń} i do
\textit{Planowania wizyt}. Wszystkie opisane są poniżej.

\paragraph{\textit{Przyjmowanie zgłoszeń}}

\paragraph{Submission}
Reprezentuje zgłoszenie na kolędę. Zawiera wszystkie niezbędne informacje o
parafianinie, takie jak dane kontaktowe oraz adres zamieszkania (w powiązanych
klasach).

Najważniejsze pola klasy \texttt{Submission} to:
\begin{itemize}
      \item \texttt{SubmitterNotes} — dodatkowe uwagi (od zgłaszającego do administratora),
      \item \texttt{AdminMessage} — informacja systemowa (od administratora do zgłaszającego),
      \item \texttt{AdminNotes} — wewnętrzne notatki (widoczne tylko dla zalogowanych użytkowników),
      \item \texttt{NotesStatus} — status realizacji dodatkowych uwag (np.\ oczekujące, zrealizowane).
\end{itemize}

\paragraph{Submitter}
Reprezentuje parafianina, który składa zgłoszenie na kolędę. Jest bezpośrednio
powiązany ze zgłoszeniem (lub wieloma).

Najważniejsze pola klasy \texttt{Submitter} to:
\begin{itemize}
      \item \texttt{Name} — imię zgłaszającego,
      \item \texttt{Surname} — nazwisko zgłaszającego,
      \item \texttt{PhoneNumber} — numer telefonu zgłaszającego,
      \item \texttt{Email} — adres e-mail zgłaszającego.
\end{itemize}

\paragraph{Address}
Reprezentuje adres zamieszkania parafianina. Jest bezpośrednio powiązany ze
zgłoszeniem (w relacji \textit{1 do 1} z dokładnością do konkretnego planu
kolędy). Jest on prawie w całości zatomizowany, aby ujednolicić format adresów,
podawanych w formularzu zgłoszeniowym (by zachować jednoznaczność).

Jest powiązany z szeregiem klas pomocniczych, które reprezentują poszczególne
elementy adresu (np.\ budynek (brama), ulica, miasto).

Najważniejsze pola klasy \texttt{Address} to:
\begin{itemize}
      \item \texttt{ApartmentNumber} — numer mieszkania,
      \item \texttt{ApartmentLetter} — litera mieszkania (opcjonalnie, raczej rzadko),
      \item właściwości \textit{cache} — kopia tekstowa właściwości z klas pomocniczych
            (dla przyspieszenia operacji bazodanowych),
      \item \texttt{FilterableString} — znormalizowany łańcuch znaków do celów filtrowania
            i wyszukiwania adresów (\textit{computed} — obliczony przez bazę danych na
            podstawie pól \textit{cache}).
\end{itemize}

\subparagraph{Building}
Reprezentuje budynek (bramę) w adresie zamieszkania parafianina. Każdy budynek
może mieć wiele adresów (mieszkań). Jest tworzony przez administratora parafii
i wybierany z listy w formularzu zgłoszeniowym (i kilku innych miejscach).

Najważniejsze pola klasy \texttt{Building} to:
\begin{itemize}
      \item \texttt{Number} — numer budynku,
      \item \texttt{Letter} — litera budynku (opcjonalnie),
      \item \texttt{FloorCount} — liczba pięter w budynku (opcjonalnie, do celów
            statystycznych),
      \item \texttt{ApartmentCount} — liczba mieszkań w budynku (opcjonalnie,
            do celów statystycznych),
      \item \texttt{HighestApartmentNumber} — najwyższy numer mieszkania w
            budynku (opcjonalnie, do celów statystycznych),
      \item \texttt{HasElevator} — czy budynek posiada windę (może mieć wpływ na
            planowanie wizyt),
      \item \texttt{AllowSelection} — czy budynek może być wybrany w formularzu
            zgłoszeniowym (w przeciwnym przypadku jest ukryty i może zostać wybrany
            tylko przez zalogowanego użytkownika).
\end{itemize}

\subparagraph{Street}
Reprezentuje ulicę w adresie zamieszkania parafianina. Każda ulica może mieć
wiele budynków. Podobnie jak budynek, jest tworzona przez administratora
parafii i wybierany z listy w formularzu zgłoszeniowym (i kilku innych
miejscach).

Najważniejsze pola klasy \texttt{Street} to:
\begin{itemize}
      \item \texttt{Name} — nazwa ulicy (bez tytułów typu ulica, aleja, plac itd.),
      \item \texttt{PostalCode} — kod pocztowy ulicy (opcjonalnie).
\end{itemize}

\subparagraph{StreetSpecifier}
Reprezentuje typ ulicy (np.\ ulica, aleja, plac itd.). Każda ulica jest
powiązana z jednym \texttt{StreetSpecifier}, który określa jej typ. Podobnie
jak budynek i ulica, jest tworzony przez administratora parafii.

Najważniejsze pola klasy \texttt{StreetSpecifier} to:
\begin{itemize}
      \item \texttt{FullName} — pełna nazwa typu ulicy (np.\ ulica, aleja, plac itd.),
      \item \texttt{Abbreviation} — skrócona nazwa typu ulicy (np.\ ul., al., pl.).
\end{itemize}

\subparagraph{City}
Reprezentuje miasto w adresie zamieszkania parafianina. Każde miasto może mieć
wiele ulic. Podobnie jak powyższe klasy, jest tworzone przez administratora
parafii.

Najważniejsze pola klasy \texttt{City} to:
\begin{itemize}
      \item \texttt{Name} — pełna nazwa miasta,
      \item \texttt{DisplayName} — wyświetlana nazwa miasta.
\end{itemize}

\paragraph{\textit{Planowanie wizyt}}

\begin{figure}[h]
      \centering
      \includegraphics[width=\textwidth]{figures/Model obiektowy - parafia.png}
      \caption{Model obiektowy aplikacji (kontekst parafialny)}
      \label{fig:model_obiektowy_parafialny}
\end{figure}

\paragraph{Visit}
Jest odpowiednikiem zgłoszenia w kontekście planowania wizyt. Reprezentuje
konkretną wizytę księdza u parafianina. Jest powiązana z danym zgłoszeniem w
relacji \textit{1 do 1}, w ten sposób są ze sobą ściśle powiązane.

Najważniejsze pola klasy \texttt{Visit} to:
\begin{itemize}
      \item \texttt{OrdinalNumber} — liczba porządkowa wizyty w agendzie,
      \item \texttt{Status} — status wizyty (np.\ zaplanowana, zaakceptowana,
            odrzucona),
      \item \texttt{PeopleCount} — liczba domowników uczestniczących w wizycie
            (do celów statystycznych).
\end{itemize}

\paragraph{Plan}
Reprezentuje plan wizyt kolędowych (najczęściej na dany rok). Zawiera wiele
dni, które z kolei zawierają agendy i wizyty. Jest zarządzany przez
parafialnego administratora aplikacji. Można przypisać do niego księży, którzy
będą w nim występować.

Najważniejsze pola klasy \texttt{Plan} to:
\begin{itemize}
      \item \texttt{Name} — nazwa planu,
      \item \texttt{CreationTime} — czas utworzenia planu.
\end{itemize}

\paragraph{Schedule}
Reprezentuje harmonogram wizyt w planie kolędy. Służy do podziału zgłoszeń
parafian na różne grupy, np.\ na kolędę w terminie i kolędę dodatkową. Każdy
plan musi zawierać przynajmniej jeden harmonogram.

Najważniejsze pola klasy \texttt{Schedule} to:
\begin{itemize}
      \item \texttt{Name} — nazwa harmonogramu,
      \item \texttt{ShortName} — skrócone oznaczenie harmonogramu,
      \item \texttt{Color} — kolor harmonogramu (zapisany w formacie szesnastkowym
            z poprzedzającym znakiem \#).
\end{itemize}

\paragraph{Day}
Reprezentuje konkretny dzień w trakcie trwania kolędy. Zawiera wiele agend,
które grupują wizyty danego dnia.

Najważniejsze pola klasy \texttt{Day} to:
\begin{itemize}
      \item \texttt{Date} — data dnia,
      \item \texttt{StartHour} — godzina rozpoczęcia kolędy w danym dniu,
      \item \texttt{EndHour} — godzina zakończenia kolędy w danym dniu.
\end{itemize}

\paragraph{Agenda}
Reprezentuje listę wizyt w konkretnym dniu kolędy. Każda agenda jest przypisana
do konkretnego dnia i można do niej przypisywać księdza oraz ministrantów.
Porządek wizyt w agendzie jest wyznaczany przez liczbę porządkową wizyty.

Najważniejsze pola klasy \texttt{Agenda} to:
\begin{itemize}
      \item \texttt{StartHourOverride} — godzina rozpoczęcia kolędy
            dla danej agendy (opcjonalnie, nadpisuje godzinę z dnia),
      \item \texttt{EndHourOverride} — godzina zakończenia kolędy
            dla danej agendy (opcjonalnie, nadpisuje godzinę z dnia),
      \item \texttt{GatheredFunds} — suma zebranych funduszy podczas wizyt
            w danej agendzie,
      \item \texttt{HideVisits} — czy ukryć wizyty w danej agendzie przed
            parafianami (np.\ przed publikacją planu, przy tworzeniu szkicu),
      \item \texttt{ShowHours} — czy pokazywać parafianom przewidywane godziny
            ich wizyty (dopiero gdy plan jest zatwierdzony).
      \item \texttt{IsOfficial} — czy pokazywać i liczyć agendę w statystykach jako pełnoprawną
            (wizyty w niej liczą się bez względu na tę właściwość).
\end{itemize}

\paragraph{BuildingAssignment}
Reprezentuje przypisanie konkretnego budynku do konkretnego dnia w planie
kolędy (w ramach wybranego harmonogramu). Pozwala to na automatyczne
sugerowanie terminów wizyt dla każdego zgłoszenia (na podstawie adresu oraz
harmonogramu), a także automatyczny zapis przy rejestracji zgłoszenia.

Jedyne pole klasy \texttt{BuildingAssignment} to:
\begin{itemize}
      \item \texttt{EnableAutoAssign} — czy włączyć automatyczne przypisywanie
            wizyt dla danego budynku w danym dniu (dla danego budynku w danym
            harmonogramie tylko jeden dzień może mieć włączony auto-zapis).
\end{itemize}

\paragraph{\textit{Tworzenie historii zmian}}

Oprócz powyższych dwóch grup w modelu znajduje się również pięć klas
pomocniczych, służących do rejestrowania historii zmian dla wybranych encji
modelu. Trzy z nich to klasy typu \textit{snapshot}:
\textbf{SubmissionSnapshot}, \textbf{SubmitterSnapshot} oraz
\textbf{VisitSnapshot}. Przechowują one kopię stanu odpowiadającej im encji w
momencie dokonania zmiany wraz z informacją o autorze zmiany. Czwarta klasa to
\textbf{FormSubmission}, która zachowuje kopię oryginalnych danych zgłoszenia
(w momencie wprowadzenia do systemu). Pozwala na odtworzenie pierwotnej wersji
zgłoszenia w przypadku nadużyć lub problemów. Ostatnia klasa to
\textbf{EmailLog}, która rejestruje wysyłane wiadomości e-mail z aplikacji,
wraz z ich zawartością i odbiorcami. Pozwala to na audyt i śledzenie
komunikacji prowadzonej przez system, a także na ponawianie próby wysyłania
wiadomości (gdy serwer pocztowy był niedostępny itp.).

\paragraph{\textit{Zarządzanie parafią}}

W modelu znajdują się również dwie encje do zarządzania informacjami dot.\
parafii: \textbf{ParishMember} oraz \textbf{ParishInfo}.

Pierwsza z nich reprezentuje użytkownika systemu, odpowiadającego dokładnie
jednemu użytkownikowi z kontekstu centralnego (klasa \texttt{User}) poprzez
pole \texttt{CentralUserId}. Podczas gdy do celów zarządzania użytkownikami i
pobierania ich informacji służy klasa \texttt{User} w kontekście centralnym, to
klasa \texttt{ParishMember} pozwala na tworzenie relacji między użytkownikami
(reprezentowanymi przez nią w kontekście parafii) a innymi encjami, np.\ planem
kolędy (przypisanie księży do planu) lub agendą (przypisanie ministrantów do
agendy).

Druga z nich jest prostym magazynem dla wszelakich informacji (w tym tych o
parafii), które mogą być potrzebne w aplikacji, ale nie mają dedykowanej encji
w modelu. Składa się z par klucz-wartość, gdzie klucz jest unikalnym
identyfikatorem informacji, a wartość przechowuje jej treść.

\paragraph{\textit{Typy wyliczeniowe}}

W modelu zastosowano również kilka typów wyliczeniowych (\textit{enum}'ów),
które służą do definiowania stałych wartości dla określonych właściwości encji.

\subparagraph{SubmitMethod}
Określa metodę, za pomocą której parafianin złożył zgłoszenie na kolędę.
Dostępne wartości to:
\begin{itemize}
      \item \textbf{NotRegistered} — nie zarejestrowano (domyślnie),
      \item \textbf{PaperForm} — formularz papierowy,
      \item \textbf{WebForm} — formularz internetowy,
      \item \textbf{Phone} — telefonicznie,
      \item \textbf{Stationary} — osobiście (np.\ w kancelarii parafialnej),
      \item \textbf{Email} — mailowo,
      \item \textbf{DuringVisit} — osobiście podczas sąsiednich wizyt danego dnia.
\end{itemize}

\subparagraph{NotesFulfillmentStatus}
Określa status realizacji dodatkowych uwag przez personel parafii. Dostępne
wartości to:
\begin{itemize}
      \item \textbf{NA} — nie dotyczy (domyślnie, gdy brak uwag),
      \item \textbf{Pending} — oczekujące (domyślnie gdy są uwagi),
      \item \textbf{Rejected} — niezrealizowane,
      \item \textbf{Accepted} — zrealizowane.
\end{itemize}

\subparagraph{VisitStatus}
Określa status wizyty kolędowej. Dostępne wartości to:
\begin{itemize}
      \item \textbf{Unplanned} — niezaplanowana (domyślnie),
      \item \textbf{Planned} — zaplanowana (w agendzie),
      \item \textbf{Pending} — trwająca,
      \item \textbf{Visited} — zrealizowana (odbyta),
      \item \textbf{Rejected} — nieodbyta (np.\ parafianin nie otworzył drzwi),
      \item \textbf{Withdrawn} — wycofana (np.\ parafianin wycofał zgłoszenie),
      \item \textbf{Suspended} — wstrzymana (w szczególnych okolicznościach,
            stan tymczasowy podczas przeprowadzania wizyt).
\end{itemize}

\section{Architektura}

\subsection{Poziom I — diagram kontekstowy systemu}

Architektura aplikacji została zaprezentowana poniżej z użyciem modelu C4. Jego
najwyższy poziom (\textit{System Context Diagram}, rys.
\ref{fig:c4_system_context_diagram}) przedstawia ogólny widok na system oraz
jego interakcje z użytkownikami i zewnętrznymi systemami.

\begin{figure}[h]
      \centering
      \includegraphics[width=\textwidth]{figures/SYS_Diagram SOK.png}
      \caption{Diagram kontekstowy systemu}
      \label{fig:c4_system_context_diagram}
\end{figure}

W naszym przypadku głównym komponentem jest aplikacja webowa \textbf{SOK}.
Umożliwia ona zarządzanie kolędą w parafiach poprzez interfejs użytkownika
dostępny z poziomu przeglądarki internetowej. Aplikacja komunikuje się
dodatkowo z serwerem pocztowym (do wysyłania powiadomień e-mail).

Na diagramie zaznaczono również dwóch aktorów: \textbf{Parafianina} oraz
\textbf{Personel parafialny}. Parafianin może składać zgłoszenia na kolędę oraz
przeglądać swój panel zgłoszenia (wszystko jako anonimowy użytkownik). Personel
parafialny zarządza zgłoszeniami i planuje wizyty kolędowe.

\subsection{Poziom II — diagram kontenerów}

Drugi poziom modelu C4 (\textit{Container Diagram}, rys.
\ref{fig:c4_container_diagram}) przedstawia główne kontenery aplikacji oraz ich
interakcje.

\begin{figure}[h]
      \centering
      \includegraphics[width=\textwidth]{figures/CON_Diagram SOK.png}
      \caption{Diagram kontenerów aplikacji}
      \label{fig:c4_container_diagram}
\end{figure}

Na tym poziomie widzimy rozróżnienie między personelem a parafianinami.
Personel, odwiedzając aplikację webową, korzysta z pełnego interfejsu
użytkownika, który udostępnia wszystkie funkcje aplikacji (w granicach roli).
Parafianin natomiast korzysta z \textit{widoków publicznych}, które są dostępne
bez logowania. W dalszej części pracy odnosząc się do \textit{widoku} będziemy
mieli na myśli interfejs użytkownika zalogowanego.

Widok aplikacji webowej może również korzystać z API, które udostępnia wybrane
funkcje aplikacji w formie usług sieciowych (por.\ rozdział
\ref{sec:vue_opis}). Aplikacja API, tak samo jak główny komponent aplikacji,
komunikuje się z bazą danych oraz z serwerem pocztowym.

\subsection{Poziom III — diagram komponentów}

\subsubsection{Aplikacja webowa}

\begin{figure}[h!]
      \centering
      \includegraphics[width=\textwidth]{figures/COM_Diagram WebApp.png}
      \caption{Diagram komponentów aplikacji webowej}
      \label{fig:c4_component_diagram_web_app}
\end{figure}

Aplikacja webowa jest zaprogramowana w architekturze MVC
(Model-View-Controller). Podział ten widoczny jest na diagramie (rys.
\ref{fig:c4_component_diagram_web_app}), choć nie jest wyraźny przy pierwszej
analizie z powodu dodatkowych komponentów, które zostały zaprezentowane ze
względu na ich istotną rolę w aplikacji.

\paragraph{Kontrolery}

Pierwszym rodzajem komponentu jest \textbf{Kontroler}, który obsługuje żądania
HTTP od użytkowników i koordynuje przepływ danych między modelem a widokiem. W
aplikacji znajdują się dwa rodzaje kontrolerów: kontrolery aplikacji (dla
personelu parafialnego) oraz kontrolery publiczne (dla parafian).

Pierwszy z nich obsługuje wszystkie funkcje aplikacji dostępne dla zalogowanych
użytkowników (w granicach ich ról). Korzysta przy tym z komponentu do
autoryzacji, aby sprawdzić uprawnienia użytkownika przed wykonaniem danej
akcji. Drugi z nich obsługuje funkcje dostępne bez logowania, tj.\ składanie
zgłoszeń na kolędę oraz przeglądanie panelu zgłoszenia.

\paragraph{Widoki}

Kolejnym rodzajem komponentu są \textbf{Widoki}, które zawierają interfejs
użytkownika aplikacji, prezentowany w przeglądarce. Widoki są tworzone przy
użyciu technologii Razor, która umożliwia dynamiczne generowanie stron HTML na
podstawie danych z modelu. Podobnie jak kontrolery, podzielone są na dwie
kategorie: widoki dla zalogowanych użytkowników oraz widoki publiczne dla
parafian.

Podczas gdy widoki publiczne są prostsze i bardziej statyczne (głównie
formularze i strony informacyjne), widoki dla zalogowanych użytkowników są
często bardziej rozbudowane i interaktywne, oferując zaawansowane funkcje
zarządzania kolędą. W sekcji \ref{sec:frontend_opis} opisano podejścia
wykorzystane do tworzenia interfejsu użytkownika w widokach aplikacji webowej.

\paragraph{Modele}

W architekturze MVC modele są przekazywane z kontrolerów do widoków, aby
prezentować dane użytkownikowi. W opisywanej aplikacji jest to każdorazowo
realizowane za pomocą jednego ze specjalnych typów modeli:

\begin{itemize}
      \item \textbf{ViewModels} (\textit{modele widoków}) — specjalne modele
            zaprojektowane do reprezentowania danych w widokach. Mogą zawierać
            dodatkową logikę prezentacyjną lub formatowanie danych.
      \item \textbf{DTO (Data Transfer Objects)} — proste obiekty, pochodzące
            bezpośrednio z warstwy aplikacji (por.\ sekcja \ref{sec:warstwa_aplikacji}),
            które często są wystarczające do prezentacji danych.
      \item \textbf{Encje EF Core} — bezpośrednie modele danych z warstwy
            domeny. Stosowane głównie w prostych widokach, gdzie nie jest wymagana
            dodatkowa logika prezentacyjna.
\end{itemize}

\paragraph{Logika biznesowa}

Logika biznesowa aplikacji jest zaimplementowana w warstwie aplikacji (por.\
sekcja \ref{sec:warstwa_aplikacji}) i jest wykorzystywana przez kontrolery
aplikacji przy użyciu \textbf{Serwisów aplikacji}, reprezentowanych na
diagramie (rys. \ref{fig:c4_component_diagram_web_app}) przez komponent
\textit{Usługi}. Serwisy aplikacji zawierają metody do wykonywania operacji
biznesowych, takich jak zarządzanie zgłoszeniami, planowanie wizyt czy
wysyłanie powiadomień e-mail. Kontrolery aplikacji nigdy nie komunikują się
bezpośrednio z warstwą domeny lub infrastrukturą, lecz zawsze poprzez serwisy
aplikacji.

\paragraph{Dostęp do danych}

Dostęp do danych w aplikacji jest realizowany za pomocą wzorca
\textbf{Repozytoriów}, które są reprezentowane na diagramie przez komponent
\textit{Repozytoria}. Repozytoria zapewniają abstrakcję nad warstwą dostępu do
danych, umożliwiając kontrolerom i serwisom aplikacji interakcję z bazą danych
bez konieczności bezpośredniego korzystania z ORM (Entity Framework Core).
Repozytoria zawierają metody do wykonywania operacji CRUD (tworzenie, odczyt,
aktualizacja, usuwanie) na encjach domeny.

Repozytoria również są zebrane w jednym komponencie. W tym przypadku jednak ma
to dodatkowy wymiar, ponieważ wszystkie one są dostępne poprzez jeden obiekt
\texttt{UnitOfWork}, który dodatkowo koordynuje transakcje i zapewnia spójność
danych.

\subsubsection{Aplikacja API}

\begin{figure}[htbp]
      \centering
      \includegraphics[width=\textwidth]{figures/COM_Diagram ApiApp.png}
      \caption{Diagram komponentów aplikacji API}
      \label{fig:c4_component_diagram_api_app}
\end{figure}

Aplikacja API działa w podobny sposób jak aplikacja webowa, jednak jej głównym
celem jest udostępnianie funkcji aplikacji w formie usług sieciowych
(dostępnych poprzez protokół HTTP). Diagram komponentów aplikacji API został
zaprezentowany na rys. \ref{fig:c4_component_diagram_api_app}.

Podobnie jak w aplikacji webowej, głównymi komponentami są tutaj
\textbf{Kontrolery}, \textbf{Serwisy aplikacji} oraz \textbf{Repozytoria}.
Kontrolery API obsługują żądania HTTP od klientów API (widoków aplikacji) i
wykonują operacje biznesowe za pomocą serwisów aplikacji. Serwisy aplikacji i
repozytoria działają identycznie jak w aplikacji webowej, zapewniając
abstrakcję nad logiką biznesową i dostępem do danych. Jedyną istotną różnicą
jest brak części publicznej, ponieważ aplikacja API jest przeznaczona wyłącznie
do użytku przez widoki aplikacji webowej dla zalogowanych użytkowników.

Aplikacja API udostępnia jedynie wybrane funkcje aplikacji, które bezpośrednio
wspierają interfejs użytkownika. Z tego powodu nie wszystkie serwisy aplikacji
i repozytoria są dostępne poprzez API.\@ Również punkty końcowe często są
dostępne tylko poprzez niektóre metody HTTP (spośród GET, POST, PUT, PATCH,
DELETE).

\section{DDD i warstwy systemu}
\label{sec:architektura_warstwowa}

Aplikację zaprojektowano i zaimplementowano w podejściu Domain-Driven Design
(DDD), tworząc warstwową architekturę, która składa się z następujących
poziomów:

\begin{itemize}
      \item Warstwa domeny (Domain Layer)
      \item Warstwa aplikacji (Application Layer)
      \item Warstwa infrastruktury (Infrastructure Layer)
      \item Warstwa interfejsu użytkownika (UI Layer)
\end{itemize}

Każdy z nich jest odpowiedzialny za określone aspekty aplikacji i komunikuje
się z innymi warstwami w sposób jasno zdefiniowany.

\subsection{Warstwa domeny}
\label{sec:warstwa_domeny}

Najniższą warstwą w architekturze jest \textbf{Warstwa domeny}, która zawiera
wszystkie elementy związane z modelem domeny aplikacji. Umieszczone zostały
tutaj głównie encje domenowe (por. \ref{sec:model_obiektowy}). Warstwa domeny
jest niezależna od innych warstw i nie zawiera żadnych odwołań do technologii
zewnętrznych (np.\ baz danych, frameworków webowych itp.).

\subsection{Warstwa aplikacji}
\label{sec:warstwa_aplikacji}

\textbf{Warstwa aplikacji} znajduje się powyżej warstwy domeny i jest odpowiedzialna
za implementację logiki biznesowej aplikacji. Zawiera serwisy aplikacji, które
realizują operacje biznesowe, korzystając z encji domeny. Warstwa aplikacji
komunikuje się z warstwą domeny poprzez bezpośrednie odwołania do encji oraz z
warstwą infrastruktury poprzez repozytoria. Jest wykorzystywana przez warstwę
interfejsu użytkownika i najczęściej przekazuje do niej dane w formie DTO (Data
Transfer Objects).

\subsection{Warstwa infrastruktury}
\label{sec:warstwa_infrastruktury}

\textbf{Warstwa infrastruktury} zapewnia implementację techniczną dla aplikacji.
Zawiera realizację repozytoriów, które umożliwiają dostęp do bazy danych
oraz inne komponenty infrastrukturalne, w tym komponent do wysyłania
wiadomości e-mail czy komponent autoryzacji. Główną odpowiedzialnością
warstwy infrastruktury jest izolowanie pozostałych warstw od szczegółów technicznych,
takich jak konkretna baza danych czy protokoły komunikacyjne.

\subsection{Warstwa UI}
\label{sec:warstwa_ui}

Najwyższym poziomem w architekturze jest \textbf{Warstwa interfejsu
      użytkownika}, która zawiera komponenty odpowiedzialne za interakcję z
użytkownikami. W naszym przypadku są to aplikacja webowa oraz aplikacja API.\@
Warstwa UI komunikuje się z warstwą aplikacji, korzystając z serwisów aplikacji
i przekazuje dane do prezentacji użytkownikowi do widoków.

\section{Architektura wdrażania}
\label{sec:architektura_wdrazania}

Jak wspomniano w sekcji \ref{sec:ci_cd_opis_technologii} aplikacja została
zaopatrzona w stosowny plik \texttt{Dockerfile}, który pozwala na zbudowanie
obrazu aplikacji. Na jego podstawie można następnie tworzyć kontenery Dockera.

W projekcie umieszczono również plik \texttt{docker-compose.yml} (wraz z
kilkoma odmianami), który pozwala na uruchomienie całego środowiska aplikacji
(z aplikacją, bazą danych i menadżerem ruchu) za pomocą jednego polecenia z
uwzględnieniem opcji konfiguracyjnych za pomocą orkiestratora Docker Compose.

\begin{figure}[h]
      \centering
      \includegraphics[width=\textwidth]{figures/DPL_Diagram SOK.png}
      \caption{Diagram wdrażania aplikacji}
      \label{fig:deployment_diagram}
\end{figure}

Na diagramie wdrażania (rys. \ref{fig:deployment_diagram}) zaprezentowano
strukturę środowiska aplikacji. Zaznaczone są na nim trzy wyżej wymienione
usługi: aplikacja webowa, baza danych oraz menadżer ruchu (reverse proxy).

\subsubsection{Konteneryzacja}

Wszystkie trzy usługi są uruchamiane w oddzielnych kontenerach Dockera, co
pozwala na ich izolację i łatwe zarządzanie. Konteneryzacja umożliwia również
łatwe skalowanie aplikacji oraz przenoszenie jej między różnymi środowiskami
(deweloperskim, testowym, produkcyjnym).

Przy zmianach w kodzie aplikacji lub jej konfiguracji wystarczy więc zbudować
nowy obraz Dockera i uruchomić nowy kontener. Zachowuje się w ten sposób
izolację od bazy danych, która nie odnotowuje żadnych przerw w działaniu.

\subsubsection{Zarządzanie siecią}

W środowisku aplikacji zastosowano również wirtualną sieć Dockera, która
pozwala na zachowanie bezpośredniej komunikacji między kontenerami przy pełnej
ich izolacji od sieci zewnętrznej. Kontenery mogą się ze sobą komunikować za
pomocą nazw usług (np.\ \texttt{database} dla bazy danych), co upraszcza
konfigurację połączeń.

Jedynym punktem dostępu do aplikacji z zewnątrz jest menadżer ruchu, który
przekazuje żądania do aplikacji webowej. Pozwala to na dodatkowe zabezpieczenie
aplikacji oraz na łatwe zarządzanie ruchem sieciowym (np.\ poprzez konfigurację
certyfikatów SSL/TLS).

\subsubsection{Trwałość danych}

Baza danych jest skonfigurowana z użyciem wolumenu Dockera, który zapewnia
trwałość danych pomiędzy restartami kontenera. Oznacza to, że dane
przechowywane w bazie danych nie zostaną utracone w przypadku ponownego
uruchomienia kontenera lub aktualizacji aplikacji. Wolumen jest mapowany na
lokalny katalog na hoście, co pozwala na łatwe tworzenie kopii zapasowych i
zarządzanie danymi bazy.

\subsubsection{Środowisko produkcyjne}

Dzięki zastosowaniu powyższego zestawu technologii i podejść architektonicznych
aplikacja jest łatwa do wdrożenia i utrzymania na dowolnym serwerze
obsługującym Dockera, w szczególności na popularnych platformach chmurowych.
Nie jest jednak do nich ograniczone — cały stos technologiczny może być
uruchomiony na dowolnym serwerze fizycznym lub wirtualnym, do którego
użytkownik ma dostęp, za pomocą jednego polecenia (bez konieczności manualnego
konfigurowania środowiska — instalowania oprogramowania i jego zależności).

\section{Opis rozwiazania wybranych problemów }
\todo{Napisać}

\subsection{Dostęp do bazy danych}

W aplikacji ze względu na wielodomenowość (wzorzec \textit{multi-tenant})
zastosowano izolację danych na poziomie bazy danych. Oznacza to, że każda
parafia posiada oddzielną bazę danych, co zapewnia pełną separację danych
między parafiami i zwiększa bezpieczeństwo. Aby umożliwić aplikacji dostęp do
odpowiedniej bazy danych w zależności od parafii, zastosowano dynamiczne
tworzenie ciągów połączeń w kontekście \textit{EF Core}.

\subsubsection{Określenie parafii}

Pierwszym elementem mechanizmu uzyskiwania dostępu do odpowiedniej bazy danych
jest \textit{middleware} \texttt{ParishResolver}, umieszczone w potoku
przetwarzania. W nim pobierane są informacje o parafii (z ciasteczka lub
parametru adresu URL).

\subsubsection{Pobranie informacji o parafii}

Następnie \texttt{ParishResolver} korzysta z serwisu
\texttt{ICurrentParishService}, który umożliwia zapisywanie i pobieranie
informacji o aktualnie wybranej parafii w trakcie trwania żądania HTTP\@.
Serwis ten ma zasięg \textit{scoped}, dzięki czemu jego stan jest zachowywany
tylko w trakcie przetwarzania danego żądania.

Główną funkcją serwisu jest pobranie informacji o parafii z centralnej bazy
danych, odszyfrowanie ciągu połączenia i zapisanie go w swoim stanie. W ten
sposób pozostaje on dostępny dla pozostałych komponentów aplikacji w trakcie
przetwarzania żądania.

\subsubsection{Utworzenie kontekstu}

Po wykonaniu powyższych kroków następuje utworzenie kontekstu bazy danych
\texttt{ParishDbContext} w kontenerze \textit{DI}\@. Wtedy wywoływana jest
nadpisana metoda \texttt{OnConfiguring}, w której pobierany jest ciąg
połączenia z serwisu \texttt{ICurrentParishService} i konfigurowany jest
kontekst do korzystania z odpowiedniej bazy danych.

Dalej w aplikacji wszystkie operacje na bazie danych są wykonywane za pomocą
kontekstu \texttt{ParishDbContext}, który jest już poprawnie skonfigurowany do
komunikacji z bazą danych odpowiedniej parafii.

\subsection{Zarządzanie parafiami}

Ze względu na izolację danych na poziomie bazy danych, tworzenie nowej parafii
wymaga utworzenia nowej bazy danych oraz odpowiedniej jej konfiguracji,
włączając utworzenie użytkownika bazy i wykonanie wszystkich należnych
migracji. Aby zautomatyzować ten proces, w aplikacji zaimplementowano serwis
\texttt{IParishProvisioningService}, który wykonuje wszystkie niezbędne kroki.
Posiada on dwie metody publiczne: \texttt{CreateParishAsync} oraz
\texttt{EnsureAllParishDatabasesReadyAsync}.

Pierwsza z nich służy do utworzenia bazy danych na poziomie serwera baz danych
i jest wywoływana przy rejestracji nowej parafii. Druga z nich jest
wykorzystywana podczas uruchamiania aplikacji, sprawdzając, czy wszystkie bazy
danych parafii (zapisanych w centralnej bazie danych) są poprawnie
skonfigurowane i gotowe do użycia. Obie metody korzystają ze wspólnej metody
prywatnej, która działa w następujących krokach:

\begin{itemize}
      \item Połączenie z serwerem baz danych przy użyciu konta administracyjnego,
      \item Utworzenie nowej bazy danych o unikalnej nazwie (z losowym sufiksem),
      \item Utworzenie nowego użytkownika z unikalną nazwą (z losowym sufiksem) oraz
            przypisanie mu odpowiednich uprawnień do tej bazy danych,
      \item Wykonanie wszystkich (nowych) migracji Entity Framework Core na tej bazie
            danych,
      \item Zapisanie informacji o parafii w centralnej bazie danych, w tym zaszyfrowanego
            ciągu połączenia do tej bazy danych.
\end{itemize}

Gdy metoda ma utworzyć parafię, wszystkie kroki wykonają się bezwarunkowo, gdy
jednak służy do sprawdzenia istnienia bazy danych, to pomija adekwatne kroki w
razie braku konieczności ich wykonania.

W ten sposób proces tworzenia parafii jest w pełni zautomatyzowany i nie wymaga
ręcznej interwencji administratora bazy danych. Używa przy tym niskopoziomowych
poleceń SQL do zarządzania bazami danych i użytkownikami, co zapewnia pełną
kontrolę nad procesem.

\chapter{Porównanie z innymi implementacjami}

Obecnie rynek aplikacji do organizacji wizyt duszpasterskich nie jest wcale
rozwinięty. Parafie, które decydują się na przeprowadzanie wizyty
duszpasterskiej w modelu kolędy na zaproszenie, najczęściej wybierają
rozwiązania albo prowizoryczne, albo prywatne, tworzone i nadzorowane przez
parafian na własną rękę. Nie istnieją jeszcze żadne aplikacje komercyjne
dedykowane do tego celu.

\section{System manualny}

Najmniej technicznym rozwiązaniem jest podejście manualne, polegające na
prowadzeniu listy wizyt w formie papierowej. Często dopuszcza również
zgłoszenia internetowe (np. poprzez formularz Google), jednak ostateczna lista
wizyt jest tworzona ręcznie przez duszpasterza lub inną osobę odpowiedzialną za
organizację kolędy na papierze, czy to poprzez układanie karteczek z nazwiskami
na stole (w dużych parafiach po całych pomieszczeniach), czy też poprzez
wypisywanie listy na tablicy. Takie podejście jest jednak bardzo niewygodne i
podatne na błędy. W przypadku dużej liczby zgłoszeń łatwo o przeoczenie lub
podwójne zapisanie wizyty. Ponadto, w przypadku konieczności zmiany terminu
wizyty, cała lista musi zostać zmodyfikowana ręcznie, co prowadzi do
dodatkowych komplikacji. Ostatecznie oprócz czasu i wysiłku zajmuje to dużo
przestrzeni fizycznej, a lista papierowa może zostać łatwo zgubiona lub
uszkodzona.

\section{System półautomatyczny}

Niektóre parafie decydują się na wykorzystanie arkuszy kalkulacyjnych (np.
Microsoft Excel lub Google Sheets) do zarządzania listą wizyt duszpasterskich.
W tym podejściu lista wizyt jest tworzona i modyfikowana cyfrowo, co pozwala na
łatwiejsze wprowadzanie zmian i zmniejsza ryzyko błędów związanych z ręcznym
zapisywaniem. Arkusze kalkulacyjne oferują funkcje sortowania i filtrowania
danych, co ułatwia organizację wizyt według różnych kryteriów, takich jak data
czy ulica. Jednakże, mimo że podejście to jest bardziej zaawansowane niż
prowadzenie listy na papierze, nadal wymaga ręcznego wprowadzania danych i
aktualizacji listy. Ponadto, arkusze kalkulacyjne nie oferują dedykowanych
funkcji do zarządzania wizytami duszpasterskimi, co może prowadzić do
nieefektywności i komplikacji w organizacji kolędy.

\section{\textit{Adventus}}

W bieżącym roku (2025) w parafii w Ujanowicach (archidiecezja krakowska)
została wdrożona aplikacja webowa \textit{Adventus} (\cite{adventusapp2026}),
stworzona przez zespół młodzieżowych programistów z tej parafii. Aplikacja ta
umożliwia parafianom śledzenie zaplanowanej trasy kolędowej na interaktywnej
mapie. Jest ona publicznie dostępna, jednak w celu sprawdzenia szczegółów
wizyty (takich jak data, wyznaczony duszpasterz itp.) wymagana jest
autentykacja parafianina poprzez podanie adresu i nazwiska rodziny.

\begin{figure}[ht]
    \centering
    \includegraphics[width=0.8\textwidth]{figures/screenshots/EXT_Adventus.png}
    \caption{Mapa trasy kolędowej w aplikacji \textit{Adventus}}
\end{figure}

Ze względu na zamknięty charakter aplikacji oraz brak dostępu do jej kodu
źródłowego, nie jest wiadome, czy posiada ona funkcje takie jak planowanie
wizyt w sposób w pełni cyfrowy, czy też wymaga ręcznego przenoszenia danych do
innego systemu przez administratora. Nie jest też jasne, czy aplikacja oferuje
takie funkcjonalności jak:

\begin{itemize}
    \item Przewidywanie godzin wizyty,
    \item Powiadamianie mailowe,
    \item Zarządzanie harmonogramami.
\end{itemize}

W systemie funkcjonuje jednak oznaczanie statusów wizyt na bieżąco przez
ministrantów podczas trwania kolędy, co jest źródłem aktualnych informacji dla
parafian (\cite{pytanienasniadanie2026}).

Główną różnicą między systemem \textit{Adventus} a opisaną w tej pracy
aplikacją jest to, że \textit{Adventus} skupia się głównie na udostępnianiu
informacji parafianom, podczas gdy aplikacja opisana w tej pracy ma na celu
ułatwienie organizacji i zarządzania wizytami duszpasterskimi od strony
administracyjnej, co w efekcie poprawia również doświadczenie parafian.

Warto też zwrócić uwagę na różnicę w podejściu do autentykacji użytkowników. W
aplikacji \textit{Adventus} parafianie muszą podać swoje dane (adres i
nazwisko) w celu uzyskania dostępu do szczegółów wizyty, co może budzić obawy
dotyczące prywatności i bezpieczeństwa danych, zwłaszcza w małych
miejscowościach, gdzie wszyscy się znają. Dodatkowo publicznie udostępnione są
informacje o trasie kolędowej, co może prowadzić do różnych niepożądanych, a
czasem i niebezpiecznych sytuacji (szczególnie w obszarach o większym
zagęszczeniu ludności).

\chapter{Instrukcja użytkownika}

Instrukcja opisuje podstawowe funkcje aplikacji oraz sposób jej obsługi przez
administratora. Wszystkie zrzuty ekranów przedstawione w instrukcji z aplikacji
uruchomionej w środowisku deweloperskim z domyślnymi zmiennymi środowiskowymi
(w szczególności z przykładowymi danymi bazy danych).

\subsection*{Logowanie}

Po uruchomieniu aplikacji użytkownik zostaje przekierowany na stronę logowania
(rys.~\ref{fig:login_page}). W celu zalogowania się należy podać nazwę
użytkownika oraz hasło i kliknąć przycisk \textit{Zaloguj się}.

\begin{figure}[h]
    \centering
    \begin{subfigure}{.5\textwidth}
        \centering
        \includegraphics[width=.95\linewidth]{figures/screenshots/SOK_Login_light.png}
        \caption{Wariant jasny}
        \label{fig:login_page_light}
    \end{subfigure}%
    \begin{subfigure}{.5\textwidth}
        \centering
        \includegraphics[width=.95\linewidth]{figures/screenshots/SOK_Login_dark.png}
        \caption{Wariant ciemny}
        \label{fig:login_page_dark}
    \end{subfigure}
    \caption{Strona logowania}
    \label{fig:login_page}
\end{figure}

W zależności od ustawień przeglądarki, aplikacja może być wyświetlana w
wariancie jasnym (rys.~\ref{fig:login_page_light}) lub ciemnym
(rys.~\ref{fig:login_page_dark}). W dalszej części instrukcji pokazywany jest
wariant jasny. W celu zmiany wariantu kolorystycznego można skorzystać z
przycisku zmiany motywu dostępnego w rozwijanym menu użytkownika w prawym
górnym rogu ekranu po zalogowaniu (należy kliknąć na nazwę użytkownika).

Przy pierwszym uruchomieniu aplikacji dostępne jest tylko konto administratora
o danych logowania zgodnych z podanymi w pliku konfiguracyjnym \texttt{.env}
(por. sekcja~\ref{sec:developer:run}). Domyślnie są to:

\begin{itemize}
    \item Nazwa użytkownika: \texttt{admin}
    \item Hasło: \texttt{admin}
\end{itemize}

\subsection*{Zarządzanie parafiami}

Z racji logowania na konto superadministratora w pierwotnym stanie aplikacji po
zalogowaniu zostanie wyświetlona strona zarządzania parafiami
(rys.~\ref{fig:manage_parishes}). Superadministrator nie jest przypisany do
żadnej parafii, lecz może załadować się do dowolnej parafii w tym widoku.

\begin{figure}[h]
    \centering
    \includegraphics[width=.75\linewidth]{figures/screenshots/SOK_Parishes_list.png}
    \caption{Zarządzanie parafiami}
    \label{fig:manage_parishes}
\end{figure}

Dostępna jest lista wszystkich parafii zarejestrowanych w systemie. Domyślnie
utworzona została przykładowa parafia z przypisanym do niej jednym
użytkownikiem — administratorem. Aby załadować się do danej parafii, należy
kliknąć przycisk \textit{Przełącz} w jej kafelku\footnote{Spowoduje to wydanie
    ciasteczka z identyfikatorem parafii, co pozwoli na dostęp do zasobów tej
    parafii po zalogowaniu. Po usunięciu danych aplikacji i restarcie może być
    potrzebne usunięcie ciasteczek w przeglądarce dla zapewnienia poprawnego
    działania.}.

W tym widoku dostępny jest też przycisk \textit{Utwórz nową parafię}, który
przekierowuje do formularza tworzenia nowej parafii. Jego przesłanie spowoduje
dodanie wpisu do listy parafii oraz utworzenie nowej bazy danych dla niej.

\subsection*{Strona główna}

Normalny użytkownik po zalogowaniu zostaje przekierowany na stronę główną
aplikacji. Jako superadministrator można również do niej wejść przez opcję
\textit{Przegląd} w menu bocznym (oznaczoną przez ikonkę domu), pod warunkiem
uprzedniego wybrania parafii. W zależności od roli użytkownika, na stronie
głównej dostępne są różne opcje menu.

\begin{figure}[h]
    \centering
    \begin{subfigure}{.33\textwidth}
        \centering
        \includegraphics[width=.95\linewidth]{figures/screenshots/SOK_Home_blank.png}
        \caption{Pierwsze wejście}
        \label{fig:home_page:blank}
    \end{subfigure}%
    \begin{subfigure}{.33\textwidth}
        \centering
        \includegraphics[width=.95\linewidth]{figures/screenshots/SOK_Home_admin.png}
        \caption{Widok administratora}
        \label{fig:home_page:admin}
    \end{subfigure}%
    \begin{subfigure}{.33\textwidth}
        \centering
        \includegraphics[width=.95\linewidth]{figures/screenshots/SOK_Home_support.png}
        \caption{Widok ministranta}
        \label{fig:home_page:support}
    \end{subfigure}
    \caption{Strona główna}
    \label{fig:home_page}
\end{figure}

Przy pierwszym wejściu nie ma żadnego aktywnego planu, a więc na stronie
głównej widoczny jest tylko komunikat z przekierowaniem
(rys.~\ref{fig:home_page:blank}).

Po utworzeniu i aktywowaniu planu na stronie głównej pojawią się kafelki z
podsumowaniem najbliższego dnia kolędowego, listą wszystkich dni kolędowych
oraz ze statystykami zgłoszeń (rys.~\ref{fig:home_page:admin}).

W widoku ministranta (rys.~\ref{fig:home_page:support}) zamiast tych kafelków
wyświetlają się kafelki z listą przydzielonych mu nadchodzących agend oraz
możliwością szybkiego przejścia do przeprowadzania wizyt.

\subsection*{Zarządzanie użytkownikami}

Aby przejść do zarządzania użytkownikami, należy w bocznym menu wybrać opcję
\textit{Ustawienia} (oznaczoną przez ikonę koła zębatego), a następnie zakładkę
\textit{Użytkownicy}. Wyświetli się wtedy strona z listą wszystkich
użytkowników (rys.~\ref{fig:manage_users:list} pokazuje stan aplikacji z
większą liczbą użytkowników).

\begin{figure}[h]
    \centering
    \includegraphics[width=.75\linewidth]{figures/screenshots/SOK_Users_list.png}
    \caption{Lista użytkowników}
    \label{fig:manage_users:list}
\end{figure}

Po kliknięciu przycisku \textit{Nowe konto} zostanie wyświetlony formularz
tworzenia nowego użytkownika (rys.~\ref{fig:manage_users:create}). W celu
edycji istniejącego użytkownika należy kliknąć na niego. Spowoduje to
wyświetlenie formularza edycji użytkownika (rys.~\ref{fig:manage_users:edit}).

\begin{figure}[h]
    \centering
    \begin{subfigure}{.5\textwidth}
        \centering
        \includegraphics[width=.95\linewidth]{figures/screenshots/SOK_Users_create.png}
        \caption{Tworzenie użytkownika}
        \label{fig:manage_users:create}
    \end{subfigure}%
    \begin{subfigure}{.5\textwidth}
        \centering
        \includegraphics[width=.95\linewidth]{figures/screenshots/SOK_Users_edit.png}
        \caption{Edycja użytkownika}
        \label{fig:manage_users:edit}
    \end{subfigure}
    \caption{Zarządzanie użytkownikami}
    \label{fig:manage_users}
\end{figure}

\subsection*{Zarządzanie planami}

Aby przejść do zarządzania planami, należy w bocznym menu wybrać opcję
\textit{Plany} (oznaczoną przez ikonę trzech poziomych kresek kaskadowych).
Wyświetli się wtedy strona z możliwością utworzenia nowego planu oraz listą
wszystkich planów (rys.~\ref{fig:manage_plans:list}).

\begin{figure}[h!]
    \centering
    \begin{subfigure}{.5\textwidth}
        \centering
        \includegraphics[width=.95\linewidth]{figures/screenshots/SOK_Plans_list.png}
        \caption{Lista planów}
        \label{fig:manage_plans:list}
    \end{subfigure}%
    \begin{subfigure}{.5\textwidth}
        \centering
        \includegraphics[width=.95\linewidth]{figures/screenshots/SOK_Plans_edit.png}
        \caption{Edycja planu}
        \label{fig:manage_plans:edit}
    \end{subfigure}
    \caption{Zarządzanie planami}
    \label{fig:manage_plans}
\end{figure}

Aby aktywować dany plan, należy kliknąć przycisk \textit{Aktywuj} w jego
wierszu na liście planów. Spowoduje to, że plan stanie się aktualnym planem
używanym przez aplikację, a wszelkie zgłoszenia będą wpływać na niego.

W głównej części widoku dostępne są szczegóły aktywnego planu, przycisk edycji
oraz możliwość przełączania przyjmowania zgłoszeń przez formularz publiczny.
Jest on zablokowany, jeśli plan nie ma wybranego domyślnego harmonogramu
(wówczas i tak nie przyjmuje zgłoszeń). Po prawej stronie prezentowane są
podstawowe statystyki planu oraz przyciski z odnośnikami do widoków z bardziej
szczegółowymi statystykami.

\begin{wrapfigure}[15]{r}{.45\textwidth}
    \centering
    \begin{subfigure}{.9\linewidth}
        \centering
        \includegraphics[width=.95\linewidth]{figures/screenshots/SOK_Plans_default_schedule.png}
        \caption{Wybór harmonogramu domyślnego}
        \label{fig:manage_plans:default_schedule}
    \end{subfigure}
    \begin{subfigure}{.9\linewidth}
        \centering
        \includegraphics[width=.95\linewidth]{figures/screenshots/SOK_Plans_schedule_modal.png}
        \caption{Edycja harmonogramu}
        \label{fig:manage_plans:schedule_modal}
    \end{subfigure}
    \caption{Zarządzanie planami — szczegóły}
    \label{fig:manage_plans:details}
\end{wrapfigure}

Aby utworzyć nowy plan, należy kliknąć przycisk \textit{Utwórz} (dostępny gdy
nie ma aktywnego planu), co spowoduje wyświetlenie formularza tworzenia planu.
W celu edycji istniejącego planu należy kliknąć na jego nazwę na liście.
Spowoduje to wyświetlenie formularza edycji planu
(rys.~\ref{fig:manage_plans:edit}).

Formularze tworzenia i edycji planu wyglądają identycznie. W formularzu należy
podać nazwę planu oraz dostępne harmonogramy i wybrać harmonogram domyślny
(rys.~\ref{fig:manage_plans:default_schedule}) i przypisanych księży. Jedynie w
miejscu można zarządzać harmonogramami planu (dodawać, edytować i usuwać
harmonogramy — rys.~\ref{fig:manage_plans:schedule_modal}).

\subsection*{Przyjmowanie zgłoszeń}

W aplikacji istnieją dwa sposoby rejestracji zgłoszeń:
\begin{itemize}
    \item poprzez publiczny formularz zgłoszeniowy,
    \item poprzez panel administratora.
\end{itemize}

\paragraph{Publiczny formularz zgłoszeniowy}
dostępny jest pod adresem:
\begin{verbatim}
    https://<adres_aplikacji>/<uid_parafii>/submissions/new
\end{verbatim}

W powyższym \texttt{<adres\_aplikacji>} to adres, pod którym dostępna jest
aplikacja, a \texttt{<uid\_parafii>} to unikalny identyfikator parafii, który
jest generowany losowo przy tworzeniu parafii, a dostępny w
\textit{Ustawieniach} w zakładce \textit{Ustawienia ogólne}.

Aby ułatwić dostęp do formularza dla personelu parafialnego, można przejść do
niego przechodząc do opcji menu bocznego \textit{Ustawienia}, a następnie
wybierając zakładkę \textit{Formularz zgłoszeniowy}.

W formularzu zgłoszeniowym (rys.~\ref{fig:public_submission_form}) należy podać
dane zgłaszającego — imię, nazwisko, adres e-mail (opcjonalnie) oraz adres.
Jeśli został skonfigurowany automatyczny zapis, to wyświetli się tam informacja
o dacie kolędy, na którą zgłaszający zostanie zapisany po wysłaniu zgłoszenia
(jeśli jest dostępna dla danego adresu).

W formularzu dostępne są do wyboru tylko bramy, które zostały wcześniej
utworzone w systemie (patrz sekcja \textit{Zarządzanie adresami}) i są bramami
widocznymi.

\begin{figure}[h]
    \centering
    \begin{subfigure}{.5\textwidth}
        \centering
        \includegraphics[width=.95\linewidth]{figures/screenshots/SOK_Public_submission_form.png}
        \caption{Publiczny formularz zgłoszeniowy}
        \label{fig:public_submission_form}
    \end{subfigure}%
    \begin{subfigure}{.5\textwidth}
        \centering
        \includegraphics[width=.95\linewidth]{figures/screenshots/SOK_Internal_submission_form.png}
        \caption{Wewnętrzny formularz zgłoszeniowy}
        \label{fig:internal_submission_form}
    \end{subfigure}
    \caption{Formularze zgłoszeniowe}
    \label{fig:submission_forms}
\end{figure}

\paragraph{Wewnętrzny formularz zgłoszeniowy}
dostępny jest w opcji menu bocznego \textit{Utwórz zgłoszenie} (oznaczoną przez
ikonę plusa w przerywanym kwadracie). W nim oprócz podstawowych informacji
można od razu uzupełnić dodatkowe dane zgłoszenia, takie wewnętrzne notatki,
numer telefonu, sposób złożenia zgłoszenia itp.
(rys.~\ref{fig:internal_submission_form}).

\subsection*{Edycja zgłoszeń}

Po rejestracji zgłoszenia można je znaleźć na liście zgłoszeń dostępnej w opcji
menu bocznego \textit{Zgłoszenia} (oznaczoną przez ikonę koperty z wystającą
kartką). W tym widoku (rys.~\ref{fig:submission:list}) po lewej stronie ekranu
dostępne są filtry umożliwiające zawężenie wyników wyszukiwania, a po prawej
stronie znajduje się lista zgłoszeń spełniających kryteria wyszukiwania i
sortowania.

\begin{figure}[h]
    \centering
    \includegraphics[width=.75\linewidth]{figures/screenshots/SOK_Submissions_list.png}
    \caption{Lista zgłoszeń}
    \label{fig:submission:list}
\end{figure}

Kliknięcie na zgłoszenie spowoduje wyświetlenie szczegółów zgłoszenia
(rys.~\ref{fig:submission:details}). W tym widoku można zapoznać się ze
wszystkimi danymi zgłoszenia oraz je edytować poprzez kliknięcie przycisku
\textit{Edytuj} w prawym górnym rogu okienka. W trybie edycji
(rys.~\ref{fig:submission:edit}) można zmieniać wszystkie dane zgłoszenia oraz
zapisać zmiany poprzez kliknięcie przycisku \textit{Zapisz}. Jeśli wówczas
podany jest adres e-mail, to na górnym pasku wyświetla się również przełącznik
wysyłania powiadomienia e-mail o zmianie danych zgłoszenia do zgłaszającego.
Podgląd wysyłanej wiadomości można zobaczyć klikając przycisk obok przełącznika
z ikonką oka (lub w menu rozwijanym w prawym górnym rogu okienka) — o ile
wprowadzone zostały zmiany, kwalifikujące się do wysłania powiadomienia.

\begin{figure}[h]
    \centering
    \begin{subfigure}{.5\textwidth}
        \centering
        \includegraphics[width=.95\linewidth]{figures/screenshots/SOK_Submission_details.png}
        \caption{Szczegóły zgłoszenia}
        \label{fig:submission:details}
    \end{subfigure}%
    \begin{subfigure}{.5\textwidth}
        \centering
        \includegraphics[width=.95\linewidth]{figures/screenshots/SOK_Submission_edit.png}
        \caption{Edycja zgłoszenia}
        \label{fig:submission:edit}
    \end{subfigure}
    \caption{Zarządzanie zgłoszeniami}
    \label{fig:submissions:manage}
\end{figure}

W menu rozwijanym w prawym górnym rogu okienka są dostępne także inne opcje,
takie jak wysyłanie powiadomień e-mail do zgłaszającego (jeśli podany jest
adres e-mail), przejście do panelu zgłoszenia (dostępnego dla zgłaszającego)
lub dostęp do oryginalnych danych zgłoszenia (przechwyconych w momencie
rejestracji zgłoszenia).

W kafelku wizyty dostępne jest również rozwijane menu (w prawym górnym rogu), w
którym jest opcja anulowania wizyty/zgłoszenia (wówczas zgłoszenie staje się
niezaplanowane, przestaje być wyświetlane w większości miejsc, a na liście
zgłoszeń ma odpowiednie oznaczenie), lub przywrócenia anulowanego zgłoszenia.

Jeśli zgłoszenie jest zaplanowane w jakiejś agendzie, to w kafelku wizyty
dostępny jest przycisk \textit{Przejdź do agendy}, który przenosi do edytora
tej agendy.

\subsection*{Zarządzanie dniami kolędowymi}

Aby przejść do zarządzania dniami kolędowymi, należy w bocznym menu wybrać
opcję \textit{Kalendarz} (oznaczoną przez ikonę kalendarza). Jeśli w danym
planie nie są jeszcze skonfigurowane daty rozpoczęcia i zakończenia kolędy, to
zostanie wyświetlony formularz ich dodania (rys.~\ref{fig:manage_days:create}).
W przeciwnym wypadku pokazana zostanie lista wszystkich dni kolędowych
(rys.~\ref{fig:manage_days:list}).

\begin{figure}[h]
    \centering
    \begin{subfigure}{.5\textwidth}
        \centering
        \includegraphics[width=.95\linewidth]{figures/screenshots/SOK_Days_create.png}
        \caption{Tworzenie i edycja dni kolędowych}
        \label{fig:manage_days:create}
    \end{subfigure}%
    \begin{subfigure}{.5\textwidth}
        \centering
        \includegraphics[width=.95\linewidth]{figures/screenshots/SOK_Days_list.png}
        \caption{Lista dni kolędowych}
        \label{fig:manage_days:list}
    \end{subfigure}
    \caption{Zarządzanie dniami kolędowymi}
    \label{fig:manage_days}
\end{figure}

Podczas dodawania lub edycji dni kolędowych można ustawić daty rozpoczęcia i
zakończenia kolędy, które potem będą umożliwiały dodanie wybranych dni z ich
zakresu (często będzie to wiele dni, są to jednak daty graniczne). Dla każdego
dnia ponadto można ustawić domyślne godziny rozpoczęcia i zakończenia wizyt,
które potem mogą być nadpisywane na poziomie poszczególnych agend.

W widoku wyświetlania dni kolędowych można przejść do widoku danego dnia przez
kliknięcie na niego. Spowoduje to wyświetlenie szczegółów dnia kolędowego
(rys.~\ref{fig:manage_days:details}), w którym dostępna jest statystyka dnia,
lista wszystkich agend oraz lista wszystkich przypisań.

\begin{figure}[h]
    \centering
    \includegraphics[width=.75\linewidth]{figures/screenshots/SOK_Day_details.png}
    \caption{Szczegóły dnia kolędowego}
    \label{fig:manage_days:details}
\end{figure}

\subsection*{Zarządzanie przypisaniami}

Sekcja przypisań prezentuje listę wszystkich przypisanych bram dla danego dnia.
Jeśli wszystkie bramy na danej ulicy są przypisane, to zamiast wyświetlać je
wszystkie system pokazuje nazwę ulicy w sekcji przypisań całych ulic.

\begin{figure}[h]
    \centering
    \begin{subfigure}{.5\textwidth}
        \centering
        \includegraphics[width=.95\linewidth]{figures/screenshots/SOK_Assignments_list.png}
        \caption{Lista przypisań}
        \label{fig:manage_assignments:list}
    \end{subfigure}%
    \begin{subfigure}{.5\textwidth}
        \centering
        \includegraphics[width=.95\linewidth]{figures/screenshots/SOK_Assignments_add.png}
        \caption{Dodawanie przypisań}
        \label{fig:manage_assignments:add}
    \end{subfigure}
    \caption{Zarządzanie przypisaniami}
    \label{fig:manage_assignments}
\end{figure}

Tworzenie, edycja i usuwanie przypisań jest możliwe po kliknięciu przycisku
\textit{Zarządzaj przypisaniami} w sekcji przypisań. Spowoduje to wyświetlenie
okienka zarządzania przypisań (rys.~\ref{fig:manage_assignments:list}).
Dostępne są w nim opcje filtrowania istniejących przypisań, usuwania ich, a
także włączania i wyłączania autozapisu dla poszczególnych przypisań. Aby dodać
nowe przypisanie, należy kliknąć przycisk \textit{Dodaj bramy}, co spowoduje
wyświetlenie okienka dodawania przypisań
(rys.~\ref{fig:manage_assignments:add}). W nim po lewej stronie dostępna jest
lista filtrów umożliwiających zawężenie listy dostępnych bram oraz podsumowanie
aktualnie wybranych bram. Po prawej stronie znajduje się lista bram,
spełniających kryteria wyszukiwania. Bramy, które są już przypisane w danym
dniu, są oznaczone i nie można ich odznaczyć.

Przypisanie danej bramy jest rozpatrywane w zasięgu jednego harmonogramu, tzn.
brama może być przypisana do tego samego dnia wiele razy w różnych
harmonogramach, ale tylko jeden raz w ramach jednego harmonogramu. Dodatkowo w
ramach jednego harmonogramu może być włączony autozapis, a w ramach drugiego
już nie.

\subsection*{Zarządzanie agendami}

W sekcji agend dostępna jest lista wszystkich agend danego dnia oraz możliwość
ich tworzenia i edycji. Kliknięcie przycisku \textit{Nowa agenda} spowoduje
wyświetlenie formularza tworzenia agendy (rys.~\ref{fig:agenda:modal}), który
jest identyczny jak formularz edycji agendy (dostępny po kliknięciu na ikonkę
ołówka w kafelku agendy).

\begin{figure}[h]
    \centering
    \includegraphics[width=.5\linewidth]{figures/screenshots/SOK_Agenda_modal.png}
    \caption{Tworzenie i edycja agendy}
    \label{fig:agenda:modal}
\end{figure}

Każda agenda ma kilka istotnych opcji konfiguracyjnych:
\begin{itemize}
    \item Godzina rozpoczęcia/zakończenia — nadpisanie opcji z dnia,
    \item Przypisany ksiądz — ksiądz, który będzie realizował listę wizyt,
    \item Przypisani ministranci — ministranci, którzy będą wspomagać księdza podczas
          wizyt,
    \item Jednostka czasowa — czas trwania pojedynczej wizyty w minutach (wliczając
          przejścia między mieszkaniami),
    \item Pokaż godziny — czy udostępniać zgłaszającym w panelu zgłoszenia przewidywany
          czas wizyty,
    \item Ukryj wizyty — czy ukryć agendę przed zgłaszającymi (wyświetlać zapisanym na
          nią informację o oczekiwaniu na zapis),
    \item Oficjalna agenda — czy agenda jest liczona do statystyk i pokazywana w
          kalendarzu (jej wizyty nadal są liczone; opcja przydatna do wizyt
          indywidualnych).
\end{itemize}

\subsection*{Planowanie wizyt}

Po kliknięciu na kafelek agendy (poza przyciskiem edycji) zostanie wyświetlony
edytor agendy (rys.~\ref{fig:agenda:editor}), w którym dostępna jest lista
wszystkich wizyt zaplanowanych w danej agendzie oraz możliwość zarządzania
nimi.

\begin{figure}[h]
    \centering
    \begin{subfigure}{.5\textwidth}
        \centering
        \includegraphics[width=.95\linewidth]{figures/screenshots/SOK_Agenda_editor.png}
        \caption{Edytor agendy}
        \label{fig:agenda:editor}
    \end{subfigure}%
    \begin{subfigure}{.5\textwidth}
        \centering
        \includegraphics[width=.95\linewidth]{figures/screenshots/SOK_Agenda_editor_add_visits.png}
        \caption{Dodawanie wizyt do agendy}
        \label{fig:agenda:editor_add_visits}
    \end{subfigure}
    \caption{Zarządzanie agendami}
    \label{fig:agendas:manage}
\end{figure}

Aby dodać wizyty, należy kliknąć przycisk \textit{Dodaj wizyty}, co spowoduje
rozwinięcie od boku okienka z listą wszystkich niezaplanowanych zgłoszeń
(rys.~\ref{fig:agenda:editor_add_visits}). W nim można skorzystać z filtrów po
lewej stronie, aby zawęzić (lub rozszerzyć) listę zgłoszeń, a następnie
zaznaczyć te, które chce się dodać do agendy. Po kliknięciu przycisku
\textit{Dodaj do agendy} zostaną one dodane do agendy jako wizyty. Domyślnie
zostaną umieszczone na końcu listy wizyt, ale jeśli system stwierdzi, że
istnieje lepsze miejsce (np.\ występuje już dana brama), to wstawi je tam.

W edytorze można zmieniać kolejność wizyt za pomocą:
\begin{itemize}
    \item przeciągania i upuszczania (\textit{drag \& drop}),
    \item przycisków przesuwania wizyty w górę lub w dół listy,
    \item przycisków przesuwania bramy w górę lub w dół listy,
    \item zaznaczania wielu sąsiadujących wizyt i przesuwania ich grupowo za pomocą
          dolnego menu.
\end{itemize}
Dla każdej wizyty dostępny jest także przycisk wyświetlania szczegółów zgłoszenia,
otwierający okno szczegółów i edycji zgłoszenia (jak w sekcji \textit{Edycja zgłoszeń}).

\subsection*{Przeprowadzanie wizyty}

W edytorze agendy dostępny jest przycisk \textit{Przeprowadź wizytę}, który
przenosi do trybu przeprowadzania wizyty (rys.~\ref{fig:agenda:conduct}). Do
tego widoku mają dostęp także przypisani do danej agendy ministranci poprzez
kafelek z listą nadchodzących agend na stronie głównej.

\begin{wrapfigure}[19]{1}{.45\textwidth}
    \centering
    \includegraphics[width=.9\linewidth]{figures/screenshots/SOK_Agenda_conduct.png}
    \caption{Przeprowadzanie wizyty}
    \label{fig:agenda:conduct}
\end{wrapfigure}

W trybie przeprowadzania wizyty dostępna jest lista wszystkich wizyt w
agendzie, dostosowana do widoku mobilnego (większe przyciski, układ pionowy) i
umożliwiająca szybkie poruszanie się po liście wizyt i oznaczanie ich statusów.
Przeprowadzanie wizyty można rozpocząć na kwadrans przed ustawionym w agendzie
(bądź w dniu) czasie rozpoczęcia wizyty. Dopiero wtedy wyświetlany jest
przycisk \textit{Rozpocznij wizytę}. Po jego kliknięciu należy wybrać księdza
(o ile nie został on wcześniej wybrany w agendzie), a następnie oznaczone
zostaje pierwsze mieszkanie.

Kafelek każdej wizyty ma różne oznaczenia kolorystyczne w zależności od jej
statusu:
\begin{itemize}
    \item zwykły — wizyta zaplanowana,
    \item zielone obramowanie — następna wizyta,
    \item żółte obramowanie — trwająca wizyta (ksiądz jest w mieszkaniu),
    \item wyszarzony kafelek — zgłoszenie odwiedzone (wizyta zakończona),
    \item czerwone tło — wizyta nieodbyta (odrzucona),
    \item żółte tło — wizyta wstrzymana (tymczasowy status).
\end{itemize}

Status wizyty wyświetlany jest także zgłaszającemu w jego panelu zgłoszenia.
Dodatkowo po rozpoczęciu wizyty przewidywane godziny są aktualizowane na
bieżąco i wyświetlane zgłaszającemu (jeśli w agendzie jest włączona opcja
pokazywania godzin).

Dolny pasek zawiera przyciski do oznaczania liczby domowników w danym
mieszkaniu oraz przyciski do kontroli całości — od lewej: oznaczenie wizyty
jako odrzuconej, dodanie mieszkania (gdy ktoś niezaplanowany poprosi o kolędę)
oraz oznaczenie wizyty jako zakończonej (odwiedzonej). Wszystkie te przyciski
odnoszą się do wizyty, oznaczonej zieloną ramką (następnej). Po oznaczeniu
wizyty jako zakończonej lub odrzuconej, automatycznie zaznaczana jest kolejna
wizyta.

Przycisk do dodawania mieszkania otwiera okienko dodawania wizyty do agendy, w
którym należy podać numer mieszkania (można dodawać tylko mieszkania z
aktualnie odwiedzanej bramy). Po dodaniu mieszkania zostaje ono od razu
oznaczone jako następne.

Więcej specjalistycznych opcji dostępnych jest w menu rozwijanym w prawym
górnym rogu ekranu (ikona trzech pionowych kropek).

\subsection*{Ustawienia aplikacji}

Aby przejść do ustawień aplikacji, należy w bocznym menu wybrać opcję
\textit{Ustawienia} (oznaczoną przez ikonę koła zębatego). Wyświetli się wtedy
strona z kilkoma zakładkami ustawień. Należy wybrać zakładkę \textit{Ustawienia
    ogólne}, aby wyświetlić listę ustawień ogólnych aplikacji dla tej parafii
(rys.~\ref{fig:settings:general}).

\begin{figure}[h]
    \centering
    \includegraphics[width=.75\linewidth]{figures/screenshots/SOK_Settings_general.png}
    \caption{Ustawienia ogólne aplikacji}
    \label{fig:settings:general}
\end{figure}

W tym miejscu można dowolnie je wszystkie zmieniać, a następnie zapisać zmiany
przyciskiem \textit{Zapisz zmiany} w przyklejonym nagłówku bądź je odrzucić
przyciskiem \textit{Cofnij zmiany} obok.

\subsection*{Zarządzanie adresami}

Po przejściu do ustawień ogólnych i wejściu w zakładkę \textit{Adresy} dostępna
jest lista wszystkich wprowadzonych do systemu bram i ulic oraz możliwość ich
tworzenia, edycji i usuwania (rys.~\ref{fig:manage_addresses:list}). Formularze
zgłoszeniowe (publiczny i wewnętrzny) pozwalają na wybór adresu jedynie spośród
wprowadzonych do systemu adresów, zatem istotne jest, aby przed rozpoczęciem
przyjmowania zgłoszeń utworzyć wszystkie potrzebne bramy w parafii.

\begin{figure}[h]
    \centering
    \begin{subfigure}{.5\textwidth}
        \centering
        \includegraphics[width=.95\linewidth]{figures/screenshots/SOK_Addresses_list.png}
        \caption{Lista adresów}
        \label{fig:manage_addresses:list}
    \end{subfigure}%
    \begin{subfigure}{.5\textwidth}
        \centering
        \includegraphics[width=.95\linewidth]{figures/screenshots/SOK_Addresses_edit.png}
        \caption{Edycja bramy}
        \label{fig:manage_addresses:edit}
    \end{subfigure}
    \caption{Zarządzanie adresami}
    \label{fig:manage_addresses}
\end{figure}

Z racji, że może to być dość czasochłonne zadanie, aplikacja umożliwia
tworzenie serii bram przy danej ulicy na podstawie podanego zakresu numerów
(rys.~\ref{fig:manage_addresses:edit}). Na przykład, aby utworzyć bramy dla
ulicy \textit{Kwiatowej} z numerami od 1 do 20, należy najpierw utworzyć ulicę
\textit{Kwiatową}, uzupełniając formularz tworzenia ulicy dostępny po
kliknięciu przycisku \textit{Nowa ulica}, a następnie rozwinąć kafelek tej
ulicy na liście ulic i kliknąć przycisk \textit{Utwórz budynek}. Spowoduje to
wyświetlenie formularza tworzenia budynku, z którego można przejść do tworzenia
bram dla podanego zakresu numerów, klikając przycisk \textit{Dodaj wiele
    budynków na raz}.

Przy tworzeniu i edycji bram dostępne są także opcje zaawansowane, przede
wszystkim określenie dostępności windy. W obecnej wersji aplikacji opcja ta
determinuje kolejność automatycznego układania mieszkań w danej bramie podczas
planowania wizyt (w bramach z windą są porządkowane malejąco).

\chapter{Dla programistów}

\section{Instrukcja instalacji}

\subsection{Środowisko deweloperskie}

\subsubsection{Wymagania}

Do uruchomienia aplikacji w środowisku deweloperskim potrzebne jest
zainstalowanie następujących narzędzi:
\begin{itemize}
    \item \texttt{docker} w wersji co najmniej 28.2.2,
    \item \texttt{docker compose} w wersji co najmniej 2.37.1,
    \item \texttt{npm} w wersji co najmniej 11.6.0 (\textit{opcjonalnie})
\end{itemize}

\subsubsection{Instalacja i uruchamianie}

Aby uruchomić aplikację w środowisku deweloperskim, należy wykonać następujące
kroki:
\begin{enumerate}
    \item Sklonować repozytorium z kodem źródłowym aplikacji:
          \begin{verbatim}
git clone https://github.com/michalchawar/SOK.NET.git
\end{verbatim}
    \item Przejść do katalogu z kodem źródłowym:
          \begin{verbatim}
cd SOK.NET
\end{verbatim}
    \item Stworzyć plik ze zmiennymi środowiskowymi na podstawie dostarczonego szablonu:
          \begin{verbatim}
cp .env.sample .env
\end{verbatim}
    \item Wypełnić plik \texttt{.env} odpowiednimi wartościami lub pozostawić domyślne
          dla środowiska deweloperskiego,
    \item Zbudować i uruchomić aplikację za pomocą narzędzia \textit{docker-compose}:
          \begin{verbatim}
docker compose up --build
\end{verbatim}
    \item Przejść pod adres \url{http://localhost:8060}, gdzie dostępna będzie aplikacja.
\end{enumerate}

\subsubsection{Rozwój}

Aby wprowadzać zmiany w kodzie źródłowym aplikacji, można użyć dowolnego
edytora kodu, należy jednak pamiętać o każdorazowym przebudowaniu obrazu
Dockera po wprowadzeniu zmian. Można to zrobić za pomocą polecenia:
\begin{verbatim}
docker compose up --build
\end{verbatim}

W celu przyspieszenia tego procesu w repozytorium dostępny jest dodatkowy plik
konfiguracyjny \texttt{docker-compose.vs-code.yml}, który umożliwia
automatyczne wykrywanie zmian w kodzie źródłowym przy pomocy polecenia
\texttt{dotnet watch}, do użytku w edytorach takich jak Visual Studio Code,
które nie obsługują technologii \textit{Hot Reload} w kontenerach Docker. Aby
uruchomić aplikację w tej konfiguracji, należy użyć polecenia:
\begin{verbatim}
docker compose -f docker-compose.yml 
               -f docker-compose.vs-code.yml up --build
\end{verbatim}
Wówczas zmiany w kodzie będą automatycznie wykrywane i aplikacja będzie
odświeżana bez konieczności ponownego budowania obrazu Dockera. Aby ponadto
umożliwić automatyczne przebudowanie arkusza stylów CSS przy zmianach w plikach
źródłowych, można uruchomić dodatkowo polecenie:
\begin{verbatim}
cd SOK.Web
npm run watch:css
\end{verbatim}
Uruchamia to usługę Tailwind CSS w trybie obserwacji zmian w plikach źródłowych
i automatycznie przebudowuje arkusz stylów CSS przy każdej zmianie. Należy
również pamiętać, że za pierwszym razem trzeba uprzednio zainstalować
zależności projektu przy pomocy polecenia:
\begin{verbatim}
npm install
\end{verbatim}

Wszystkie powyższe kroki opisane są również w pliku \texttt{README.md} w
repozytorium aplikacji.

\subsubsection{Środowisko produkcyjne}

Aby uruchomić aplikację w środowisku produkcyjnym, należy wykonać podobne kroki
jak w przypadku środowiska deweloperskiego, z tą różnicą, że należy użyć pliku
konfiguracyjnego \texttt{docker-compose.prod.yml} zamiast
\texttt{docker-compose.yml}. Polecenie uruchamiające aplikację w trybie
produkcyjnym wygląda następująco:
\begin{verbatim}
docker compose -f docker-compose.prod.yml up --build
\end{verbatim}

Należy również pamiętać o odpowiednim skonfigurowaniu zmiennych środowiskowych
w pliku \texttt{.env} pod kątem środowiska produkcyjnego.

\section{Testy jednostkowe}

Tak jak wspomniano w rozdziale \ref{chap:architecture}, aplikacja posiada testy
jednostkowe pokrywające kluczowe funkcjonalności logiki biznesowej. Testy te
znajdują się w katalogu \texttt{SOK.Tests} w repozytorium i można je uruchomić
za pomocą narzędzia \texttt{dotnet test}. Aby to zrobić, należy przejść do
katalogu z kodem źródłowym aplikacji i wykonać polecenie:
\begin{verbatim}
    dotnet test SOK.Tests
\end{verbatim}

Uruchomi ono wszystkie testy jednostkowe i wyświetli wyniki w konsoli. Testy te
obejmują wybrane kluczowe scenariusze użycia aplikacji, weryfikując poprawność
zarówno najważniejszych encji modelu dziedzinowego w warstwie domeny, jak i
najistotniejszych serwisów w warstwie aplikacji.

\section{Statystyki kodu}

Tabela \ref{tab:code_stats} przedstawia statystyki linii kodu projektu
wygenerowane za pomocą narzędzia \texttt{cloc}. Zademonstrowano podział na
języki programowania oraz typy plików, z wyszczególnieniem liczby plików,
pustych wierszy, komentarzy oraz właściwego kodu źródłowego.

\begin{table}[h!]
    \centering
    \begin{tabular}{|l|r|r|r|r|}
        \hline
        \textbf{Język}         & \textbf{Pliki} & \textbf{Puste wiersze} & \textbf{Komentarze} & \textbf{Kod}    \\ \hline
        C\#                    & 305            & 11 625                 & 4 995               & 41 126          \\ \hline
        Razor                  & 58             & 881                    & 426                 & 10 086          \\ \hline
        JSON                   & 9              & 0                      & 0                   & 4 275           \\ \hline
        HTML                   & 8              & 205                    & 133                 & 1 140           \\ \hline
        JavaScript             & 4              & 41                     & 31                  & 231             \\ \hline
        YAML                   & 3              & 8                      & 2                   & 147             \\ \hline
        CSS                    & 2              & 34                     & 8                   & 125             \\ \hline
        Visual Studio Solution & 1              & 1                      & 1                   & 106             \\ \hline
        MSBuild script         & 5              & 20                     & 0                   & 98              \\ \hline
        Markdown               & 4              & 34                     & 0                   & 81              \\ \hline
        Dockerfile             & 1              & 13                     & 11                  & 32              \\ \hline
        XML                    & 1              & 0                      & 0                   & 19              \\ \hline
        \textbf{SUMA}          & \textbf{401}   & \textbf{12 862}        & \textbf{5 607}      & \textbf{57 466} \\ \hline
    \end{tabular}
    \caption{Statystyki projektu (na podstawie narzędzia \texttt{cloc})}
    \label{tab:code_stats}
\end{table}

Analizę przeprowadzono na kodzie źródłowym aplikacji bezpośrednio po
sklonowaniu repozytorium, bez wprowadzania jakichkolwiek zmian. Wykluczono przy
tym pliki bibliotek zewnętrznych oraz pliki binarne i tymczasowe, generowane
przez środowisko uruchomieniowe.

\chapter{Podsumowanie}

\section{Osiągnięcia i wnioski}

W niniejszej pracy przedstawiono kompleksowe podejście do analizy i
implementacji systemu zarządzania danymi. Główne osiągnięcia obejmują:

\begin{itemize}
      \item szczegółową analizę wymagań i wdrożenie systemu zarządzania danymi,
            uwzględniającą aspekty skalowalności, bezpieczeństwa i wydajności,
      \item projekt architektury systemu, który umożliwia efektywne przechowywanie i
            przetwarzanie danych,
      \item implementację całej aplikacji z wykorzystaniem nowoczesnych technologii,
      \item zastosowanie najlepszych praktyk programistycznych i wzorców projektowych,
      \item opakowanie całości w prosty do utrzymania i rozwijania system.
\end{itemize}

Praca nad tym projektem pozwoliła na zdobycie cennego doświadczenia w zakresie
projektowania i implementacji systemów informatycznych, a także na zrozumienie
kluczowych aspektów zarządzania danymi w kontekście współczesnych wyzwań
technologicznych.

Pełna integracja wszystkich komponentów systemu wymagała dużego nakładu pracy i
zaangażowania, a także specjalistycznej wiedzy z różnych dziedzin informatyki.
Głównym wyzwaniem było zapewnienie, że system będzie nie tylko funkcjonalny,
ale także przystępny zarówno dla użytkowników końcowych, o często przeciętnych
zdolnościach technicznych, jak i dla administratorów odpowiedzialnych za jego
utrzymanie. Przy wypełnianiu tego celu kluczowe okazało się poznanie
wymienionych w rozdziale~\ref{sec:rozdzial_III} technologii i narzędzi oraz ich
efektywne zastosowanie w praktyce.

Aplikacja została wdrożona i przetestowana w rzeczywistych warunkach, co
pozwoliło na zweryfikowanie jej funkcjonalności i wydajności. Uzyskane wyniki
potwierdziły skuteczność zastosowanych rozwiązań i wskazały kierunki dalszego
rozwoju systemu. Należy przy tym pamiętać, że w kontekście obecnej luki na
rynku oprogramowania przeznaczonego do organizacji wizyt duszpasterskich, nawet
najprostsze rozwiązania mogą przynieść znaczące korzyści użytkownikom końcowym.
Szczególnie więc pierwsze w pełni zintegrowane, wszechstronne i ogólnodostępne
narzędzie może okazać się przełomowe w tym obszarze, wyznaczając nowe standardy
i otwierając drogę do dalszych innowacji.

\section{Dalsza praca}

\subsection{Rozwój interfejsu}

Choć aplikacja spełnia wszystkie założenia funkcjonalne, w samym modelu
obiektowym pozostawiony został potencjał do dalszej rozbudowy i optymalizacji.
W przyszłości planowane jest wprowadzenie kolejnycch funkcji, takich jak
wersjonowanie niektórych danych, integracja z innymi systemami zarządzania
danymi, czy też implementacja bardziej rozbudowanych opcji analizy i statystyk.

Uprości to jeszcze bardziej organizację kolędy oraz pozwoli na lepsze
dostosowanie systemu do indywidualnych potrzeb użytkowników. Aplikacja zyska
przez to kolejną realną przewagę nad metodami bardziej manualnymi.

\subsection{Bezpieczeństwo danych}

W trakcie implementacji aplikacji stworzono podstawowe mechanizmy
zabezpieczające dane użytkowników, takie jak uwierzytelnianie i autoryzacja, a
także szyfrowanie niektórych danych w bazie. W przyszłości należy jednak
rozważyć wdrożenie bardziej zaawansowanych mechanizmów bezpieczeństwa, takich
jak monitorowanie dostępu do danych, audyt zmian czy też implementacja
mechanizmów, umożliwiających rotację kluczy szyfrowania.

\subsection{Wzbogacenie planowania}

W przyszłych wersjach aplikacji planowane jest wprowadzenie bardziej
zaawansowanych funkcji planowania wizyt duszpasterskich. Jedną z nich będzie
możliwość interaktywnego planowania tras wizyt na mapie obszaru, co pozwoli na
optymalizację czasu i zasobów potrzebnych do realizacji wizyt. Dodatkowo, na
tej podstawie możliwe będzie również zaproponowanie algorytmów automatycznego
generowania planów wizyt, uwzględniających różne kryteria, takie jak
preferencje duszpasterzy czy specyficzne potrzeby parafian, jednocześnie
optymalizując trasę pod kątem odległości i czasu podróży.

\subsection{Kartoteki osobowe}

Potencjalnym kierunkiem rozwoju aplikacji jest wzbogacenie jej o
funkcjonalności związane z zarządzaniem kartotekami osobowymi parafian.
Umożliwiłoby to przechowywanie szczegółowych informacji o mieszkańcach parafii,
takich jak dane kontaktowe, historia wizyt duszpasterskich, preferencje czy
specjalne potrzeby, a w dalszym etapie również pełnych danych, potrzebnych do
prowadzenia dokumentacji sakramentalnej.

Jest to szczególnie obiecujący kierunek rozwoju, ponieważ okres wizyt
kolędowych w praktyce duszpasterskiej często wiąże się z aktualizacjami i
uzupełnianiem danych osobowych parafian, a współcześnie wiele parafii nie
posiada spójnego i mobilnego systemu do zarządzania tymi informacjami, a
duszpasterze często muszą polegać na papierowych kartotekach lub rozproszonych
notatkach. Integracja takiej funkcjonalności z istniejącym systemem zarządzania
wizytami duspasterskimi mogłaby znacząco usprawnić pracę duszpasterzy,
umożliwiając im łatwy dostęp do aktualnych informacji o parafianach podczas
wizyt, co z kolei przyczyniłoby się do bardziej efektywnego i
spersonalizowanego podejścia do duszpasterstwa.

\subsection{Aplikacja mobilna}

Kolejnym, najbardziej obecnie odległym krokiem w rozwoju aplikacji mogłoby być
stworzenie dedykowanej aplikacji mobilnej, która umożliwiłaby duszpasterzom i
ministrantom dostęp do systemu zarządzania wizytami duszpasterskimi
bezpośrednio z ich smartfonów lub tabletów. Taka aplikacja mogłaby oferować
funkcje podobne do tych dostępnych w wersji webowej, ale zoptymalizowane pod
kątem urządzeń mobilnych, co pozwoliłoby na jeszcze większą wygodę i
elastyczność w zarządzaniu wizytami duszpasterskimi w terenie.

Podczas gdy aplikacja webowa, dostępna przez przeglądarkę internetową, oferuje
w zupełności wystarczający zakres funkcjonalności i jest łatwo dostępna z
różnych urządzeń, aplikacja mobilna mogłaby dodać dodatkową warstwę
użytkowości, zwłaszcza w kontekście pracy duszpasterzy w terenie. Funkcje takie
jak powiadomienia push, możliwość pracy offline czy integracja z funkcjami
urządzenia mobilnego (np. GPS, aparat fotograficzny) mogłyby znacząco ulepszyć
doświadczenie użytkowników i zwiększyć efektywność zarządzania wizytami
duszpasterskimi, szczególnie w kontekście poprzedniego punktu, dotyczącego
przechowywania kartotek osobowych parafian. Z racji większej kontroli nad
urządzeniem i jego zasobami, aplikacja mobilna przede wszystkim oferowałaby
lepsze bezpieczeństwo danych, co jest kluczowe w kontekście przechowywania
wrażliwych informacji osobowych.

%%%%% BIBLIOGRAFIA

\bibliographystyle{plain}
\bibliography{bibliography}

\end{document}