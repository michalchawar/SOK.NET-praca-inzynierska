\chapter{Wprowadzenie}

\section{Problematyka}

Doroczna wizyta duszpasterska (tzw. kolęda) jest tradycyjnym elementem życia
religijnego w Polsce. Polega ona na odwiedzinach księdza w domach parafian,
podczas których udziela on błogosławieństwa, rozmawia z mieszkańcami oraz
zbiera ofiary na potrzeby parafii. Organizacja wizyt duszpasterskich może być
jednak wyzwaniem logistycznym, zwłaszcza w większych parafiach. Często wymaga
to koordynacji wielu osób, ustalania terminów oraz zarządzania informacjami o
parafianach.

W dzisiejszych czasach w zależności od parafii wizyty duszpasterskie
przeprowadzane są w dwóch modelach:

\begin{itemize}
      \item \textbf{Kolęda tradycyjna}, podczas której duszpasterze starają się
            dotrzeć do wszystkich parafian.
      \item \textbf{Kolęda na zaproszenie}, podczas której duszpasterze odwiedzają
            tylko tych parafian, którzy wcześniej samodzielnie zgłosili chęć
            przyjęcia wizyty.
\end{itemize}

W niniejszej pracy skupiono się na opracowaniu aplikacji internetowej,
wspierającej organizację wizyt duszpasterskich, przeprowadzanych w modelu
\emph{kolędy na zaproszenie}. Gdy dalej mowa o kolędzie, chodzi właśnie o ten
model wizyty duszpasterskiej.

\subsection{Kolęda tradycyjna}
Model tradycyjny stosowany jest w większości parafii w Polsce. Funkcjonuje od
dziesiątek lat i jest dobrze znany zarówno duszpasterzom, jak i parafianom. W
tym modelu księża starają się odwiedzić wszystkich parafian w określonym
czasie, zwyczajowo w okresie Bożego Narodzenia. Czynią to według ustalonego
harmonogramu, zazwyczaj prostego i niewymagającego skomplikowanej logistyki.

Podczas gdy to podejście ma wiele zalet duszpasterskich, to jednak jest mocno
czasochłonne i obciążające dla duszpasterzy oraz wiernych. Cierpi na nim często
też jakoś wizyt, zarówno z powodu bardziej ścisłego ograniczenia czasowego na
wizytę, jak i znacznego odsetka parafian, którzy wizytę przyjmują z obowiązku
lub przyzwyczajenia, a nie z potrzeby duchowej. W większych parafiach taka
forma dodatkowo angażuje ministrantów do dodatkowej pracy organizacyjnej —
zapowiadania kolędy, czyli uprzedniego chodzenia od drzwi do drzwi i
informowania o terminie wizyty oraz zbierania wstępnych deklaracji chęci
przyjęcia wizyty. Jest to również główny sposób wczesnego prognozowania liczby
wizyt danego dnia.

\subsection{Kolęda na zaproszenie}
Model ten zyskuje na popularności w ostatnich latach, zwłaszcza w większych
miastach, gdzie parafianie mogą mieć bardziej zróżnicowane potrzeby, a odsetek
odwiedzanych domów może być mniejszy. Znany jest również poza granicami Polski.
W tej formie parafianie sami zgłaszają chęć przyjęcia wizyty duszpasterskiej
poprzez odpowiedni formularz (papierowy lub elektroniczny).

Z perspektywy organizacyjnej, model ten jest znacznie bardziej efektywny.
Pozwala dokładnie planować wizyty na każdy dzień, co zmniejsza obciążenie
duszpasterzy, zwiększając jednocześnie kontrolę nad rozłożeniem wizyt w czasie
oraz ich równomierność. Likwiduje dodatkowo potrzebę uprzedniego zapowiadania
kolędy przez ministrantów, tworząc jednak nową odpowiedzialność za szczegółowe
zaplanowanie porządku na dany dzień. Tym zajmuje się albo sam duszpasterz, albo
wyznaczona do tego osoba (np. kościelny lub inny pracownik parafii) jako
koordynator wizyty duszpasterskiej.

Kolęda na zaproszenie generuje jednak całkiem nową potrzebę — skutecznego
zarządzania zgłoszeniami parafian. W większych parafiach liczba zgłoszeń może
być znaczna, co wymaga odpowiednich narzędzi do ich rejestracji, przetwarzania
i planowania wizyt. Wymaga to również odpowiedniej komunikacji z parafianami,
aby zapewnić im jasne informacje o terminach wizyt oraz ewentualnych zmianach w
harmonogramie. Dokładnie tym wymaganiom ma na celu sprostać opisywana w
niniejszej pracy aplikacja internetowa.

\section{Plan pracy}