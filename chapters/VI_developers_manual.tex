\chapter{Dla programistów}

\section{Instrukcja instalacji}

\subsection{Środowisko deweloperskie}

\subsubsection{Wymagania}

Do uruchomienia aplikacji w środowisku deweloperskim potrzebne jest
zainstalowanie następujących narzędzi:
\begin{itemize}
    \item \texttt{docker} w wersji co najmniej 28.2.2,
    \item \texttt{docker compose} w wersji co najmniej 2.37.1,
    \item \texttt{npm} w wersji co najmniej 11.6.0 (\textit{opcjonalnie})
\end{itemize}

\subsubsection{Instalacja i uruchamianie}

Aby uruchomić aplikację w środowisku deweloperskim, należy wykonać następujące
kroki:
\begin{enumerate}
    \item Sklonować repozytorium z kodem źródłowym aplikacji:
          \begin{verbatim}
git clone https://github.com/michalchawar/SOK.NET.git
\end{verbatim}
    \item Przejść do katalogu z kodem źródłowym:
          \begin{verbatim}
cd SOK.NET
\end{verbatim}
    \item Stworzyć plik ze zmiennymi środowiskowymi na podstawie dostarczonego szablonu:
          \begin{verbatim}
cp .env.sample .env
\end{verbatim}
    \item Wypełnić plik \texttt{.env} odpowiednimi wartościami,
    \item Uruchomić aplikację za pomocą Dockera:
          \begin{verbatim}
docker compose up --build
\end{verbatim}
    \item Przejść pod adres \url{http://localhost:8060}, gdzie dostępna będzie aplikacja.
\end{enumerate}

\subsubsection{Rozwój}

Aby wprowadzać zmiany w kodzie źródłowym aplikacji, można użyć dowolnego
edytora kodu, należy jednak pamiętać o każdorazowym przebudowaniu obrazu
Dockera po wprowadzeniu zmian. Można to zrobić za pomocą polecenia:
\begin{verbatim}
docker compose up --build
\end{verbatim}

W celu przyspieszenia tego procesu w repozytorium dostępny jest dodatkowy plik
konfiguracyjny \texttt{docker-compose.vs-code.yml}, który umożliwia
automatyczne wykrywanie zmian w kodzie źródłowym przy pomocy polecenia
\texttt{dotnet watch}, do użytku w edytorach takich jak Visual Studio Code,
które nie obsługują technologii \textit{Hot Reload} w kontenerach Docker. Aby
uruchomić aplikację w tej konfiguracji, należy użyć polecenia:
\begin{verbatim}
docker compose -f docker-compose.yml 
               -f docker-compose.vs-code.yml up --build
\end{verbatim}
Wówczas zmiany w kodzie będą automatycznie wykrywane i aplikacja będzie
odświeżana bez konieczności ponownego budowania obrazu Dockera. Aby ponadto
umożliwić automatyczne przebudowanie arkusza stylów CSS przy zmianach w plikach
źródłowych, można uruchomić dodatkowo polecenie:
\begin{verbatim}
cd SOK.Web
npm run watch:css
\end{verbatim}
Uruchamia to usługę Tailwind CSS w trybie obserwacji zmian w plikach źródłowych
i automatycznie przebudowuje arkusz stylów CSS przy każdej zmianie. Należy
również pamiętać, że za pierwszym razem trzeba uprzednio zainstalować
zależności projektu przy pomocy polecenia:
\begin{verbatim}
npm install
\end{verbatim}

Wszystkie powyższe kroki opisane są również w pliku \texttt{README.md} w
repozytorium aplikacji.

\subsubsection{Środowisko produkcyjne}

Aby uruchomić aplikację w środowisku produkcyjnym, należy wykonać podobne kroki
jak w przypadku środowiska deweloperskiego, z tą różnicą, że należy użyć pliku
konfiguracyjnego \texttt{docker-compose.prod.yml} zamiast
\texttt{docker-compose.yml}. Polecenie uruchamiające aplikację w trybie
produkcyjnym wygląda następująco:
\begin{verbatim}
docker compose -f docker-compose.prod.yml up --build
\end{verbatim}

Należy również pamiętać o odpowiednim skonfigurowaniu zmiennych środowiskowych
w pliku \texttt{.env} pod kątem środowiska produkcyjnego.

\todo{Lepsza obsługa kluczy szyfrujących?}

\section{Statystyki, testy jednostkowe }
\todo{Napisać}