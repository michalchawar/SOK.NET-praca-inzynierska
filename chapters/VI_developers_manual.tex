\chapter{Dla programistów}

\section{Instrukcja instalacji}

\subsection{Środowisko deweloperskie}

\subsubsection{Wymagania}

Do uruchomienia aplikacji w środowisku deweloperskim potrzebne jest
zainstalowanie następujących narzędzi:
\begin{itemize}
    \item \texttt{docker} w wersji co najmniej 28.2.2,
    \item \texttt{docker compose} w wersji co najmniej 2.37.1,
    \item \texttt{npm} w wersji co najmniej 11.6.0 (\textit{opcjonalnie})
\end{itemize}

\subsubsection{Instalacja i uruchamianie}

Aby uruchomić aplikację w środowisku deweloperskim, należy wykonać następujące
kroki:
\begin{enumerate}
    \item Sklonować repozytorium z kodem źródłowym aplikacji:
          \begin{verbatim}
git clone https://github.com/michalchawar/SOK.NET.git
\end{verbatim}
    \item Przejść do katalogu z kodem źródłowym:
          \begin{verbatim}
cd SOK.NET
\end{verbatim}
    \item Stworzyć plik ze zmiennymi środowiskowymi na podstawie dostarczonego szablonu:
          \begin{verbatim}
cp .env.sample .env
\end{verbatim}
    \item Wypełnić plik \texttt{.env} odpowiednimi wartościami lub pozostawić domyślne
          dla środowiska deweloperskiego,
    \item Zbudować i uruchomić aplikację za pomocą narzędzia \textit{docker-compose}:
          \begin{verbatim}
docker compose up --build
\end{verbatim}
    \item Przejść pod adres \url{http://localhost:8060}, gdzie dostępna będzie aplikacja.
\end{enumerate}

\subsubsection{Rozwój}

Aby wprowadzać zmiany w kodzie źródłowym aplikacji, można użyć dowolnego
edytora kodu, należy jednak pamiętać o każdorazowym przebudowaniu obrazu
Dockera po wprowadzeniu zmian. Można to zrobić za pomocą polecenia:
\begin{verbatim}
docker compose up --build
\end{verbatim}

W celu przyspieszenia tego procesu w repozytorium dostępny jest dodatkowy plik
konfiguracyjny \texttt{docker-compose.vs-code.yml}, który umożliwia
automatyczne wykrywanie zmian w kodzie źródłowym przy pomocy polecenia
\texttt{dotnet watch}, do użytku w edytorach takich jak Visual Studio Code,
które nie obsługują technologii \textit{Hot Reload} w kontenerach Docker. Aby
uruchomić aplikację w tej konfiguracji, należy użyć polecenia:
\begin{verbatim}
docker compose -f docker-compose.yml 
               -f docker-compose.vs-code.yml up --build
\end{verbatim}
Wówczas zmiany w kodzie będą automatycznie wykrywane i aplikacja będzie
odświeżana bez konieczności ponownego budowania obrazu Dockera. Aby ponadto
umożliwić automatyczne przebudowanie arkusza stylów CSS przy zmianach w plikach
źródłowych, można uruchomić dodatkowo polecenie:
\begin{verbatim}
cd SOK.Web
npm run watch:css
\end{verbatim}
Uruchamia to usługę Tailwind CSS w trybie obserwacji zmian w plikach źródłowych
i automatycznie przebudowuje arkusz stylów CSS przy każdej zmianie. Należy
również pamiętać, że za pierwszym razem trzeba uprzednio zainstalować
zależności projektu przy pomocy polecenia:
\begin{verbatim}
npm install
\end{verbatim}

Wszystkie powyższe kroki opisane są również w pliku \texttt{README.md} w
repozytorium aplikacji.

\subsubsection{Środowisko produkcyjne}

Aby uruchomić aplikację w środowisku produkcyjnym, należy wykonać podobne kroki
jak w przypadku środowiska deweloperskiego, z tą różnicą, że należy użyć pliku
konfiguracyjnego \texttt{docker-compose.prod.yml} zamiast
\texttt{docker-compose.yml}. Polecenie uruchamiające aplikację w trybie
produkcyjnym wygląda następująco:
\begin{verbatim}
docker compose -f docker-compose.prod.yml up --build
\end{verbatim}

Należy również pamiętać o odpowiednim skonfigurowaniu zmiennych środowiskowych
w pliku \texttt{.env} pod kątem środowiska produkcyjnego.

\section{Testy jednostkowe}

Tak jak wspomniano w rozdziale \ref{chap:architecture}, aplikacja posiada testy
jednostkowe pokrywające kluczowe funkcjonalności logiki biznesowej. Testy te
znajdują się w katalogu \texttt{SOK.Tests} w repozytorium i można je uruchomić
za pomocą narzędzia \texttt{dotnet test}. Aby to zrobić, należy przejść do
katalogu z kodem źródłowym aplikacji i wykonać polecenie:
\begin{verbatim}
    dotnet test SOK.Tests
\end{verbatim}

Uruchomi ono wszystkie testy jednostkowe i wyświetli wyniki w konsoli. Testy te
obejmują wybrane kluczowe scenariusze użycia aplikacji, weryfikując poprawność
zarówno najważniejszych encji modelu dziedzinowego w warstwie domeny, jak i
najistotniejszych serwisów w warstwie aplikacji.

\section{Statystyki kodu}

Tabela \ref{tab:code_stats} przedstawia statystyki linii kodu projektu
wygenerowane za pomocą narzędzia \texttt{cloc}. Zademonstrowano podział na
języki programowania oraz typy plików, z wyszczególnieniem liczby plików,
pustych wierszy, komentarzy oraz właściwego kodu źródłowego.

\begin{table}[h!]
    \centering
    \begin{tabular}{|l|r|r|r|r|}
        \hline
        \textbf{Język}         & \textbf{Pliki} & \textbf{Puste wiersze} & \textbf{Komentarze} & \textbf{Kod}    \\ \hline
        C\#                    & 305            & 11 625                 & 4 995               & 41 126          \\ \hline
        Razor                  & 58             & 881                    & 426                 & 10 086          \\ \hline
        JSON                   & 9              & 0                      & 0                   & 4 275           \\ \hline
        HTML                   & 8              & 205                    & 133                 & 1 140           \\ \hline
        JavaScript             & 4              & 41                     & 31                  & 231             \\ \hline
        YAML                   & 3              & 8                      & 2                   & 147             \\ \hline
        CSS                    & 2              & 34                     & 8                   & 125             \\ \hline
        Visual Studio Solution & 1              & 1                      & 1                   & 106             \\ \hline
        MSBuild script         & 5              & 20                     & 0                   & 98              \\ \hline
        Markdown               & 4              & 34                     & 0                   & 81              \\ \hline
        Dockerfile             & 1              & 13                     & 11                  & 32              \\ \hline
        XML                    & 1              & 0                      & 0                   & 19              \\ \hline
        \textbf{SUMA}          & \textbf{401}   & \textbf{12 862}        & \textbf{5 607}      & \textbf{57 466} \\ \hline
    \end{tabular}
    \caption{Statystyki projektu (na podstawie narzędzia \texttt{cloc})}
    \label{tab:code_stats}
\end{table}

Analizę przeprowadzono na kodzie źródłowym aplikacji bezpośrednio po
sklonowaniu repozytorium, bez wprowadzania jakichkolwiek zmian. Wykluczono przy
tym pliki bibliotek zewnętrznych oraz pliki binarne i tymczasowe, generowane
przez środowisko uruchomieniowe.