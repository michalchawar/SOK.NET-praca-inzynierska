\chapter{Podsumowanie}

\section{Osiągnięcia i wnioski}

W niniejszej pracy przedstawiono kompleksowe podejście do analizy i
implementacji systemu zarządzania danymi. Główne osiągnięcia obejmują:

\begin{itemize}
      \item szczegółową analizę wymagań i wdrożenie systemu zarządzania danymi,
            uwzględniającą aspekty skalowalności, bezpieczeństwa i wydajności,
      \item projekt architektury systemu, który umożliwia efektywne przechowywanie i
            przetwarzanie danych,
      \item implementację całej aplikacji z wykorzystaniem nowoczesnych technologii,
      \item zastosowanie najlepszych praktyk programistycznych i wzorców projektowych,
      \item opakowanie całości w prosty do utrzymania i rozwijania system.
\end{itemize}

Praca nad tym projektem pozwoliła na zdobycie cennego doświadczenia w zakresie
projektowania i implementacji systemów informatycznych, a także na zrozumienie
kluczowych aspektów zarządzania danymi w kontekście współczesnych wyzwań
technologicznych.

Pełna integracja wszystkich komponentów systemu wymagała dużego nakładu pracy i
zaangażowania, a także specjalistycznej wiedzy z różnych dziedzin informatyki.
Głównym wyzwaniem było zapewnienie, że system będzie nie tylko funkcjonalny,
ale także przystępny zarówno dla użytkowników końcowych, o często przeciętnych
zdolnościach technicznych, jak i dla administratorów odpowiedzialnych za jego
utrzymanie. Przy wypełnianiu tego celu kluczowe okazało się poznanie
wymienionych w rozdziale~\ref{sec:rozdzial_III} technologii i narzędzi oraz ich
efektywne zastosowanie w praktyce.

Aplikacja została wdrożona i przetestowana w rzeczywistych warunkach, co
pozwoliło na zweryfikowanie jej funkcjonalności i wydajności. Uzyskane wyniki
potwierdziły skuteczność zastosowanych rozwiązań i wskazały kierunki dalszego
rozwoju systemu. Należy przy tym pamiętać, że w kontekście obecnej luki na
rynku oprogramowania przeznaczonego do organizacji wizyt duszpasterskich, nawet
najprostsze rozwiązania mogą przynieść znaczące korzyści użytkownikom końcowym.
Szczególnie więc pierwsze w pełni zintegrowane, wszechstronne i ogólnodostępne
narzędzie może okazać się przełomowe w tym obszarze, wyznaczając nowe standardy
i otwierając drogę do dalszych innowacji.

\section{Dalsza praca}

\subsection{Rozwój interfejsu}

Choć aplikacja spełnia wszystkie założenia funkcjonalne, w samym modelu
obiektowym pozostawiony został potencjał do dalszej rozbudowy i optymalizacji.
W przyszłości planowane jest wprowadzenie kolejnycch funkcji, takich jak
wersjonowanie niektórych danych, integracja z innymi systemami zarządzania
danymi, czy też implementacja bardziej rozbudowanych opcji analizy i statystyk.

Uprości to jeszcze bardziej organizację kolędy oraz pozwoli na lepsze
dostosowanie systemu do indywidualnych potrzeb użytkowników. Aplikacja zyska
przez to kolejną realną przewagę nad metodami bardziej manualnymi.

\subsection{Bezpieczeństwo danych}

W trakcie implementacji aplikacji stworzono podstawowe mechanizmy
zabezpieczające dane użytkowników, takie jak uwierzytelnianie i autoryzacja, a
także szyfrowanie niektórych danych w bazie. W przyszłości należy jednak
rozważyć wdrożenie bardziej zaawansowanych mechanizmów bezpieczeństwa, takich
jak monitorowanie dostępu do danych, audyt zmian czy też implementacja
mechanizmów, umożliwiających rotację kluczy szyfrowania.

\subsection{Wzbogacenie planowania}

W przyszłych wersjach aplikacji planowane jest wprowadzenie bardziej
zaawansowanych funkcji planowania wizyt duszpasterskich. Jedną z nich będzie
możliwość interaktywnego planowania tras wizyt na mapie obszaru, co pozwoli na
optymalizację czasu i zasobów potrzebnych do realizacji wizyt. Dodatkowo, na
tej podstawie możliwe będzie również zaproponowanie algorytmów automatycznego
generowania planów wizyt, uwzględniających różne kryteria, takie jak
preferencje duszpasterzy czy specyficzne potrzeby parafian, jednocześnie
optymalizując trasę pod kątem odległości i czasu podróży.

\subsection{Kartoteki osobowe}

Potencjalnym kierunkiem rozwoju aplikacji jest wzbogacenie jej o
funkcjonalności związane z zarządzaniem kartotekami osobowymi parafian.
Umożliwiłoby to przechowywanie szczegółowych informacji o mieszkańcach parafii,
takich jak dane kontaktowe, historia wizyt duszpasterskich, preferencje czy
specjalne potrzeby, a w dalszym etapie również pełnych danych, potrzebnych do
prowadzenia dokumentacji sakramentalnej.

Jest to szczególnie obiecujący kierunek rozwoju, ponieważ okres wizyt
kolędowych w praktyce duszpasterskiej często wiąże się z aktualizacjami i
uzupełnianiem danych osobowych parafian, a współcześnie wiele parafii nie
posiada spójnego i mobilnego systemu do zarządzania tymi informacjami, a
duszpasterze często muszą polegać na papierowych kartotekach lub rozproszonych
notatkach. Integracja takiej funkcjonalności z istniejącym systemem zarządzania
wizytami duspasterskimi mogłaby znacząco usprawnić pracę duszpasterzy,
umożliwiając im łatwy dostęp do aktualnych informacji o parafianach podczas
wizyt, co z kolei przyczyniłoby się do bardziej efektywnego i
spersonalizowanego podejścia do duszpasterstwa.

\subsection{Aplikacja mobilna}

Kolejnym, najbardziej obecnie odległym krokiem w rozwoju aplikacji mogłoby być
stworzenie dedykowanej aplikacji mobilnej, która umożliwiłaby duszpasterzom i
ministrantom dostęp do systemu zarządzania wizytami duszpasterskimi
bezpośrednio z ich smartfonów lub tabletów. Taka aplikacja mogłaby oferować
funkcje podobne do tych dostępnych w wersji webowej, ale zoptymalizowane pod
kątem urządzeń mobilnych, co pozwoliłoby na jeszcze większą wygodę i
elastyczność w zarządzaniu wizytami duszpasterskimi w terenie.

Podczas gdy aplikacja webowa, dostępna przez przeglądarkę internetową, oferuje
w zupełności wystarczający zakres funkcjonalności i jest łatwo dostępna z
różnych urządzeń, aplikacja mobilna mogłaby dodać dodatkową warstwę
użytkowości, zwłaszcza w kontekście pracy duszpasterzy w terenie. Funkcje takie
jak powiadomienia push, możliwość pracy offline czy integracja z funkcjami
urządzenia mobilnego (np. GPS, aparat fotograficzny) mogłyby znacząco ulepszyć
doświadczenie użytkowników i zwiększyć efektywność zarządzania wizytami
duszpasterskimi, szczególnie w kontekście poprzedniego punktu, dotyczącego
przechowywania kartotek osobowych parafian. Z racji większej kontroli nad
urządzeniem i jego zasobami, aplikacja mobilna przede wszystkim oferowałaby
lepsze bezpieczeństwo danych, co jest kluczowe w kontekście przechowywania
wrażliwych informacji osobowych.