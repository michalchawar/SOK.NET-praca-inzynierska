\chapter{Instrukcja użytkownika}

Instrukcja opisuje podstawowe funkcje aplikacji oraz sposób jej obsługi przez
administratora. Wszystkie zrzuty ekranów przedstawione w instrukcji z aplikacji
uruchomionej w środowisku deweloperskim z domyślnymi zmiennymi środowiskowymi
(w szczególności z przykładowymi danymi bazy danych).

\subsection*{Logowanie}

Po uruchomieniu aplikacji użytkownik zostaje przekierowany na stronę logowania
(rys.~\ref{fig:login_page}). W celu zalogowania się należy podać nazwę
użytkownika oraz hasło i kliknąć przycisk \textit{Zaloguj się}.

\begin{figure}[h]
    \centering
    \begin{subfigure}{.5\textwidth}
        \centering
        \includegraphics[width=.95\linewidth]{figures/screenshots/SOK_Login_light.png}
        \caption{Wariant jasny}
        \label{fig:login_page_light}
    \end{subfigure}%
    \begin{subfigure}{.5\textwidth}
        \centering
        \includegraphics[width=.95\linewidth]{figures/screenshots/SOK_Login_dark.png}
        \caption{Wariant ciemny}
        \label{fig:login_page_dark}
    \end{subfigure}
    \caption{Strona logowania}
    \label{fig:login_page}
\end{figure}

W zależności od ustawień przeglądarki, aplikacja może być wyświetlana w
wariancie jasnym (rys.~\ref{fig:login_page_light}) lub ciemnym
(rys.~\ref{fig:login_page_dark}). W dalszej części instrukcji pokazywany jest
wariant jasny. W celu zmiany wariantu kolorystycznego można skorzystać z
przycisku zmiany motywu dostępnego w rozwijanym menu użytkownika w prawym
górnym rogu ekranu po zalogowaniu (należy kliknąć na nazwę użytkownika).

Przy pierwszym uruchomieniu aplikacji dostępne jest tylko konto administratora
o danych logowania zgodnych z podanymi w pliku konfiguracyjnym \texttt{.env}
(por. sekcja~\ref{sec:developer:run}). Domyślnie są to:

\begin{itemize}
    \item Nazwa użytkownika: \texttt{admin}
    \item Hasło: \texttt{admin}
\end{itemize}

\subsection*{Zarządzanie parafiami}

Z racji logowania na konto superadministratora w pierwotnym stanie aplikacji po
zalogowaniu zostanie wyświetlona strona zarządzania parafiami
(rys.~\ref{fig:manage_parishes}). Superadministrator nie jest przypisany do
żadnej parafii, lecz może załadować się do dowolnej parafii w tym widoku.

\begin{figure}[h]
    \centering
    \includegraphics[width=.75\linewidth]{figures/screenshots/SOK_Parishes_list.png}
    \caption{Zarządzanie parafiami}
    \label{fig:manage_parishes}
\end{figure}

Dostępna jest lista wszystkich parafii zarejestrowanych w systemie. Domyślnie
utworzona została przykładowa parafia z przypisanym do niej jednym
użytkownikiem — administratorem. Aby załadować się do danej parafii, należy
kliknąć przycisk \textit{Przełącz} w jej kafelku\footnote{Spowoduje to wydanie
    ciasteczka z identyfikatorem parafii, co pozwoli na dostęp do zasobów tej
    parafii po zalogowaniu. Po usunięciu danych aplikacji i restarcie może być
    potrzebne usunięcie ciasteczek w przeglądarce dla zapewnienia poprawnego
    działania.}.

W tym widoku dostępny jest też przycisk \textit{Utwórz nową parafię}, który
przekierowuje do formularza tworzenia nowej parafii. Jego przesłanie spowoduje
dodanie wpisu do listy parafii oraz utworzenie nowej bazy danych dla niej.

\subsection*{Strona główna}

Normalny użytkownik po zalogowaniu zostaje przekierowany na stronę główną
aplikacji. Jako superadministrator można również do niej wejść przez opcję
\textit{Przegląd} w menu bocznym (oznaczoną przez ikonkę domu), pod warunkiem
uprzedniego wybrania parafii. W zależności od roli użytkownika, na stronie
głównej dostępne są różne opcje menu.

\begin{figure}[h]
    \centering
    \begin{subfigure}{.33\textwidth}
        \centering
        \includegraphics[width=.95\linewidth]{figures/screenshots/SOK_Home_blank.png}
        \caption{Pierwsze wejście}
        \label{fig:home_page:blank}
    \end{subfigure}%
    \begin{subfigure}{.33\textwidth}
        \centering
        \includegraphics[width=.95\linewidth]{figures/screenshots/SOK_Home_admin.png}
        \caption{Widok administratora}
        \label{fig:home_page:admin}
    \end{subfigure}%
    \begin{subfigure}{.33\textwidth}
        \centering
        \includegraphics[width=.95\linewidth]{figures/screenshots/SOK_Home_support.png}
        \caption{Widok ministranta}
        \label{fig:home_page:support}
    \end{subfigure}
    \caption{Strona główna}
    \label{fig:home_page}
\end{figure}

Przy pierwszym wejściu nie ma żadnego aktywnego planu, a więc na stronie
głównej widoczny jest tylko komunikat z przekierowaniem
(rys.~\ref{fig:home_page:blank}).

Po utworzeniu i aktywowaniu planu na stronie głównej pojawią się kafelki z
podsumowaniem najbliższego dnia kolędowego, listą wszystkich dni kolędowych
oraz ze statystykami zgłoszeń (rys.~\ref{fig:home_page:admin}).

W widoku ministranta (rys.~\ref{fig:home_page:support}) zamiast tych kafelków
wyświetlają się kafelki z listą przydzielonych mu nadchodzących agend oraz
możliwością szybkiego przejścia do przeprowadzania wizyt.

\subsection*{Zarządzanie użytkownikami}

Aby przejść do zarządzania użytkownikami, należy w bocznym menu wybrać opcję
\textit{Ustawienia} (oznaczoną przez ikonę koła zębatego), a następnie zakładkę
\textit{Użytkownicy}. Wyświetli się wtedy strona z listą wszystkich
użytkowników (rys.~\ref{fig:manage_users:list} pokazuje stan aplikacji z
większą liczbą użytkowników).

\begin{figure}[h]
    \centering
    \includegraphics[width=.75\linewidth]{figures/screenshots/SOK_Users_list.png}
    \caption{Lista użytkowników}
    \label{fig:manage_users:list}
\end{figure}

Po kliknięciu przycisku \textit{Nowe konto} zostanie wyświetlony formularz
tworzenia nowego użytkownika (rys.~\ref{fig:manage_users:create}). W celu
edycji istniejącego użytkownika należy kliknąć na niego. Spowoduje to
wyświetlenie formularza edycji użytkownika (rys.~\ref{fig:manage_users:edit}).

\begin{figure}[h]
    \centering
    \begin{subfigure}{.5\textwidth}
        \centering
        \includegraphics[width=.95\linewidth]{figures/screenshots/SOK_Users_create.png}
        \caption{Tworzenie użytkownika}
        \label{fig:manage_users:create}
    \end{subfigure}%
    \begin{subfigure}{.5\textwidth}
        \centering
        \includegraphics[width=.95\linewidth]{figures/screenshots/SOK_Users_edit.png}
        \caption{Edycja użytkownika}
        \label{fig:manage_users:edit}
    \end{subfigure}
    \caption{Zarządzanie użytkownikami}
    \label{fig:manage_users}
\end{figure}

\subsection*{Zarządzanie planami}

Aby przejść do zarządzania planami, należy w bocznym menu wybrać opcję
\textit{Plany} (oznaczoną przez ikonę trzech poziomych kresek kaskadowych).
Wyświetli się wtedy strona z możliwością utworzenia nowego planu oraz listą
wszystkich planów (rys.~\ref{fig:manage_plans:list}).

\begin{figure}[h!]
    \centering
    \begin{subfigure}{.5\textwidth}
        \centering
        \includegraphics[width=.95\linewidth]{figures/screenshots/SOK_Plans_list.png}
        \caption{Lista planów}
        \label{fig:manage_plans:list}
    \end{subfigure}%
    \begin{subfigure}{.5\textwidth}
        \centering
        \includegraphics[width=.95\linewidth]{figures/screenshots/SOK_Plans_edit.png}
        \caption{Edycja planu}
        \label{fig:manage_plans:edit}
    \end{subfigure}
    \caption{Zarządzanie planami}
    \label{fig:manage_plans}
\end{figure}

Aby aktywować dany plan, należy kliknąć przycisk \textit{Aktywuj} w jego
wierszu na liście planów. Spowoduje to, że plan stanie się aktualnym planem
używanym przez aplikację, a wszelkie zgłoszenia będą wpływać na niego.

W głównej części widoku dostępne są szczegóły aktywnego planu, przycisk edycji
oraz możliwość przełączania przyjmowania zgłoszeń przez formularz publiczny.
Jest on zablokowany, jeśli plan nie ma wybranego domyślnego harmonogramu
(wówczas i tak nie przyjmuje zgłoszeń). Po prawej stronie prezentowane są
podstawowe statystyki planu oraz przyciski z odnośnikami do widoków z bardziej
szczegółowymi statystykami.

\begin{wrapfigure}[15]{r}{.45\textwidth}
    \centering
    \begin{subfigure}{.9\linewidth}
        \centering
        \includegraphics[width=.95\linewidth]{figures/screenshots/SOK_Plans_default_schedule.png}
        \caption{Wybór harmonogramu domyślnego}
        \label{fig:manage_plans:default_schedule}
    \end{subfigure}
    \begin{subfigure}{.9\linewidth}
        \centering
        \includegraphics[width=.95\linewidth]{figures/screenshots/SOK_Plans_schedule_modal.png}
        \caption{Edycja harmonogramu}
        \label{fig:manage_plans:schedule_modal}
    \end{subfigure}
    \caption{Zarządzanie planami — szczegóły}
    \label{fig:manage_plans:details}
\end{wrapfigure}

Aby utworzyć nowy plan, należy kliknąć przycisk \textit{Utwórz} (dostępny gdy
nie ma aktywnego planu), co spowoduje wyświetlenie formularza tworzenia planu.
W celu edycji istniejącego planu należy kliknąć na jego nazwę na liście.
Spowoduje to wyświetlenie formularza edycji planu
(rys.~\ref{fig:manage_plans:edit}).

Formularze tworzenia i edycji planu wyglądają identycznie. W formularzu należy
podać nazwę planu oraz dostępne harmonogramy i wybrać harmonogram domyślny
(rys.~\ref{fig:manage_plans:default_schedule}) i przypisanych księży. Jedynie w
miejscu można zarządzać harmonogramami planu (dodawać, edytować i usuwać
harmonogramy — rys.~\ref{fig:manage_plans:schedule_modal}).

\subsection*{Przyjmowanie zgłoszeń}

W aplikacji istnieją dwa sposoby rejestracji zgłoszeń:
\begin{itemize}
    \item poprzez publiczny formularz zgłoszeniowy,
    \item poprzez panel administratora.
\end{itemize}

\paragraph{Publiczny formularz zgłoszeniowy}
dostępny jest pod adresem:
\begin{verbatim}
    https://<adres_aplikacji>/<uid_parafii>/submissions/new
\end{verbatim}

W powyższym \texttt{<adres\_aplikacji>} to adres, pod którym dostępna jest
aplikacja, a \texttt{<uid\_parafii>} to unikalny identyfikator parafii, który
jest generowany losowo przy tworzeniu parafii, a dostępny w
\textit{Ustawieniach} w zakładce \textit{Ustawienia ogólne}.

Aby ułatwić dostęp do formularza dla personelu parafialnego, można przejść do
niego przechodząc do opcji menu bocznego \textit{Ustawienia}, a następnie
wybierając zakładkę \textit{Formularz zgłoszeniowy}.

W formularzu zgłoszeniowym (rys.~\ref{fig:public_submission_form}) należy podać
dane zgłaszającego — imię, nazwisko, adres e-mail (opcjonalnie) oraz adres.
Jeśli został skonfigurowany automatyczny zapis, to wyświetli się tam informacja
o dacie kolędy, na którą zgłaszający zostanie zapisany po wysłaniu zgłoszenia
(jeśli jest dostępna dla danego adresu).

W formularzu dostępne są do wyboru tylko bramy, które zostały wcześniej
utworzone w systemie (patrz sekcja \textit{Zarządzanie adresami}) i są bramami
widocznymi.

\begin{figure}[h]
    \centering
    \begin{subfigure}{.5\textwidth}
        \centering
        \includegraphics[width=.95\linewidth]{figures/screenshots/SOK_Public_submission_form.png}
        \caption{Publiczny formularz zgłoszeniowy}
        \label{fig:public_submission_form}
    \end{subfigure}%
    \begin{subfigure}{.5\textwidth}
        \centering
        \includegraphics[width=.95\linewidth]{figures/screenshots/SOK_Internal_submission_form.png}
        \caption{Wewnętrzny formularz zgłoszeniowy}
        \label{fig:internal_submission_form}
    \end{subfigure}
    \caption{Formularze zgłoszeniowe}
    \label{fig:submission_forms}
\end{figure}

\paragraph{Wewnętrzny formularz zgłoszeniowy}
dostępny jest w opcji menu bocznego \textit{Utwórz zgłoszenie} (oznaczoną przez
ikonę plusa w przerywanym kwadracie). W nim oprócz podstawowych informacji
można od razu uzupełnić dodatkowe dane zgłoszenia, takie wewnętrzne notatki,
numer telefonu, sposób złożenia zgłoszenia itp.
(rys.~\ref{fig:internal_submission_form}).

\subsection*{Edycja zgłoszeń}

Po rejestracji zgłoszenia można je znaleźć na liście zgłoszeń dostępnej w opcji
menu bocznego \textit{Zgłoszenia} (oznaczoną przez ikonę koperty z wystającą
kartką). W tym widoku (rys.~\ref{fig:submission:list}) po lewej stronie ekranu
dostępne są filtry umożliwiające zawężenie wyników wyszukiwania, a po prawej
stronie znajduje się lista zgłoszeń spełniających kryteria wyszukiwania i
sortowania.

\begin{figure}[h]
    \centering
    \includegraphics[width=.75\linewidth]{figures/screenshots/SOK_Submissions_list.png}
    \caption{Lista zgłoszeń}
    \label{fig:submission:list}
\end{figure}

Kliknięcie na zgłoszenie spowoduje wyświetlenie szczegółów zgłoszenia
(rys.~\ref{fig:submission:details}). W tym widoku można zapoznać się ze
wszystkimi danymi zgłoszenia oraz je edytować poprzez kliknięcie przycisku
\textit{Edytuj} w prawym górnym rogu okienka. W trybie edycji
(rys.~\ref{fig:submission:edit}) można zmieniać wszystkie dane zgłoszenia oraz
zapisać zmiany poprzez kliknięcie przycisku \textit{Zapisz}. Jeśli wówczas
podany jest adres e-mail, to na górnym pasku wyświetla się również przełącznik
wysyłania powiadomienia e-mail o zmianie danych zgłoszenia do zgłaszającego.
Podgląd wysyłanej wiadomości można zobaczyć klikając przycisk obok przełącznika
z ikonką oka (lub w menu rozwijanym w prawym górnym rogu okienka) — o ile
wprowadzone zostały zmiany, kwalifikujące się do wysłania powiadomienia.

\begin{figure}[h]
    \centering
    \begin{subfigure}{.5\textwidth}
        \centering
        \includegraphics[width=.95\linewidth]{figures/screenshots/SOK_Submission_details.png}
        \caption{Szczegóły zgłoszenia}
        \label{fig:submission:details}
    \end{subfigure}%
    \begin{subfigure}{.5\textwidth}
        \centering
        \includegraphics[width=.95\linewidth]{figures/screenshots/SOK_Submission_edit.png}
        \caption{Edycja zgłoszenia}
        \label{fig:submission:edit}
    \end{subfigure}
    \caption{Zarządzanie zgłoszeniami}
    \label{fig:submissions:manage}
\end{figure}

W menu rozwijanym w prawym górnym rogu okienka są dostępne także inne opcje,
takie jak wysyłanie powiadomień e-mail do zgłaszającego (jeśli podany jest
adres e-mail), przejście do panelu zgłoszenia (dostępnego dla zgłaszającego)
lub dostęp do oryginalnych danych zgłoszenia (przechwyconych w momencie
rejestracji zgłoszenia).

W kafelku wizyty dostępne jest również rozwijane menu (w prawym górnym rogu), w
którym jest opcja anulowania wizyty/zgłoszenia (wówczas zgłoszenie staje się
niezaplanowane, przestaje być wyświetlane w większości miejsc, a na liście
zgłoszeń ma odpowiednie oznaczenie), lub przywrócenia anulowanego zgłoszenia.

Jeśli zgłoszenie jest zaplanowane w jakiejś agendzie, to w kafelku wizyty
dostępny jest przycisk \textit{Przejdź do agendy}, który przenosi do edytora
tej agendy.

\subsection*{Zarządzanie dniami kolędowymi}

Aby przejść do zarządzania dniami kolędowymi, należy w bocznym menu wybrać
opcję \textit{Kalendarz} (oznaczoną przez ikonę kalendarza). Jeśli w danym
planie nie są jeszcze skonfigurowane daty rozpoczęcia i zakończenia kolędy, to
zostanie wyświetlony formularz ich dodania (rys.~\ref{fig:manage_days:create}).
W przeciwnym wypadku pokazana zostanie lista wszystkich dni kolędowych
(rys.~\ref{fig:manage_days:list}).

\begin{figure}[h]
    \centering
    \begin{subfigure}{.5\textwidth}
        \centering
        \includegraphics[width=.95\linewidth]{figures/screenshots/SOK_Days_create.png}
        \caption{Tworzenie i edycja dni kolędowych}
        \label{fig:manage_days:create}
    \end{subfigure}%
    \begin{subfigure}{.5\textwidth}
        \centering
        \includegraphics[width=.95\linewidth]{figures/screenshots/SOK_Days_list.png}
        \caption{Lista dni kolędowych}
        \label{fig:manage_days:list}
    \end{subfigure}
    \caption{Zarządzanie dniami kolędowymi}
    \label{fig:manage_days}
\end{figure}

Podczas dodawania lub edycji dni kolędowych można ustawić daty rozpoczęcia i
zakończenia kolędy, które potem będą umożliwiały dodanie wybranych dni z ich
zakresu (często będzie to wiele dni, są to jednak daty graniczne). Dla każdego
dnia ponadto można ustawić domyślne godziny rozpoczęcia i zakończenia wizyt,
które potem mogą być nadpisywane na poziomie poszczególnych agend.

W widoku wyświetlania dni kolędowych można przejść do widoku danego dnia przez
kliknięcie na niego. Spowoduje to wyświetlenie szczegółów dnia kolędowego
(rys.~\ref{fig:manage_days:details}), w którym dostępna jest statystyka dnia,
lista wszystkich agend oraz lista wszystkich przypisań.

\begin{figure}[h]
    \centering
    \includegraphics[width=.75\linewidth]{figures/screenshots/SOK_Day_details.png}
    \caption{Szczegóły dnia kolędowego}
    \label{fig:manage_days:details}
\end{figure}

\subsection*{Zarządzanie przypisaniami}

Sekcja przypisań prezentuje listę wszystkich przypisanych bram dla danego dnia.
Jeśli wszystkie bramy na danej ulicy są przypisane, to zamiast wyświetlać je
wszystkie system pokazuje nazwę ulicy w sekcji przypisań całych ulic.

\begin{figure}[h]
    \centering
    \begin{subfigure}{.5\textwidth}
        \centering
        \includegraphics[width=.95\linewidth]{figures/screenshots/SOK_Assignments_list.png}
        \caption{Lista przypisań}
        \label{fig:manage_assignments:list}
    \end{subfigure}%
    \begin{subfigure}{.5\textwidth}
        \centering
        \includegraphics[width=.95\linewidth]{figures/screenshots/SOK_Assignments_add.png}
        \caption{Dodawanie przypisań}
        \label{fig:manage_assignments:add}
    \end{subfigure}
    \caption{Zarządzanie przypisaniami}
    \label{fig:manage_assignments}
\end{figure}

Tworzenie, edycja i usuwanie przypisań jest możliwe po kliknięciu przycisku
\textit{Zarządzaj przypisaniami} w sekcji przypisań. Spowoduje to wyświetlenie
okienka zarządzania przypisań (rys.~\ref{fig:manage_assignments:list}).
Dostępne są w nim opcje filtrowania istniejących przypisań, usuwania ich, a
także włączania i wyłączania autozapisu dla poszczególnych przypisań. Aby dodać
nowe przypisanie, należy kliknąć przycisk \textit{Dodaj bramy}, co spowoduje
wyświetlenie okienka dodawania przypisań
(rys.~\ref{fig:manage_assignments:add}). W nim po lewej stronie dostępna jest
lista filtrów umożliwiających zawężenie listy dostępnych bram oraz podsumowanie
aktualnie wybranych bram. Po prawej stronie znajduje się lista bram,
spełniających kryteria wyszukiwania. Bramy, które są już przypisane w danym
dniu, są oznaczone i nie można ich odznaczyć.

Przypisanie danej bramy jest rozpatrywane w zasięgu jednego harmonogramu, tzn.
brama może być przypisana do tego samego dnia wiele razy w różnych
harmonogramach, ale tylko jeden raz w ramach jednego harmonogramu. Dodatkowo w
ramach jednego harmonogramu może być włączony autozapis, a w ramach drugiego
już nie.

\subsection*{Zarządzanie agendami}

W sekcji agend dostępna jest lista wszystkich agend danego dnia oraz możliwość
ich tworzenia i edycji. Kliknięcie przycisku \textit{Nowa agenda} spowoduje
wyświetlenie formularza tworzenia agendy (rys.~\ref{fig:agenda:modal}), który
jest identyczny jak formularz edycji agendy (dostępny po kliknięciu na ikonkę
ołówka w kafelku agendy).

\begin{figure}[h]
    \centering
    \includegraphics[width=.5\linewidth]{figures/screenshots/SOK_Agenda_modal.png}
    \caption{Tworzenie i edycja agendy}
    \label{fig:agenda:modal}
\end{figure}

Każda agenda ma kilka istotnych opcji konfiguracyjnych:
\begin{itemize}
    \item Godzina rozpoczęcia/zakończenia — nadpisanie opcji z dnia,
    \item Przypisany ksiądz — ksiądz, który będzie realizował listę wizyt,
    \item Przypisani ministranci — ministranci, którzy będą wspomagać księdza podczas
          wizyt,
    \item Jednostka czasowa — czas trwania pojedynczej wizyty w minutach (wliczając
          przejścia między mieszkaniami),
    \item Pokaż godziny — czy udostępniać zgłaszającym w panelu zgłoszenia przewidywany
          czas wizyty,
    \item Ukryj wizyty — czy ukryć agendę przed zgłaszającymi (wyświetlać zapisanym na
          nią informację o oczekiwaniu na zapis),
    \item Oficjalna agenda — czy agenda jest liczona do statystyk i pokazywana w
          kalendarzu (jej wizyty nadal są liczone; opcja przydatna do wizyt
          indywidualnych).
\end{itemize}

\subsection*{Planowanie wizyt}

Po kliknięciu na kafelek agendy (poza przyciskiem edycji) zostanie wyświetlony
edytor agendy (rys.~\ref{fig:agenda:editor}), w którym dostępna jest lista
wszystkich wizyt zaplanowanych w danej agendzie oraz możliwość zarządzania
nimi.

\begin{figure}[h]
    \centering
    \begin{subfigure}{.5\textwidth}
        \centering
        \includegraphics[width=.95\linewidth]{figures/screenshots/SOK_Agenda_editor.png}
        \caption{Edytor agendy}
        \label{fig:agenda:editor}
    \end{subfigure}%
    \begin{subfigure}{.5\textwidth}
        \centering
        \includegraphics[width=.95\linewidth]{figures/screenshots/SOK_Agenda_editor_add_visits.png}
        \caption{Dodawanie wizyt do agendy}
        \label{fig:agenda:editor_add_visits}
    \end{subfigure}
    \caption{Zarządzanie agendami}
    \label{fig:agendas:manage}
\end{figure}

Aby dodać wizyty, należy kliknąć przycisk \textit{Dodaj wizyty}, co spowoduje
rozwinięcie od boku okienka z listą wszystkich niezaplanowanych zgłoszeń
(rys.~\ref{fig:agenda:editor_add_visits}). W nim można skorzystać z filtrów po
lewej stronie, aby zawęzić (lub rozszerzyć) listę zgłoszeń, a następnie
zaznaczyć te, które chce się dodać do agendy. Po kliknięciu przycisku
\textit{Dodaj do agendy} zostaną one dodane do agendy jako wizyty. Domyślnie
zostaną umieszczone na końcu listy wizyt, ale jeśli system stwierdzi, że
istnieje lepsze miejsce (np.\ występuje już dana brama), to wstawi je tam.

W edytorze można zmieniać kolejność wizyt za pomocą:
\begin{itemize}
    \item przeciągania i upuszczania (\textit{drag \& drop}),
    \item przycisków przesuwania wizyty w górę lub w dół listy,
    \item przycisków przesuwania bramy w górę lub w dół listy,
    \item zaznaczania wielu sąsiadujących wizyt i przesuwania ich grupowo za pomocą
          dolnego menu.
\end{itemize}
Dla każdej wizyty dostępny jest także przycisk wyświetlania szczegółów zgłoszenia,
otwierający okno szczegółów i edycji zgłoszenia (jak w sekcji \textit{Edycja zgłoszeń}).

\subsection*{Przeprowadzanie wizyty}

W edytorze agendy dostępny jest przycisk \textit{Przeprowadź wizytę}, który
przenosi do trybu przeprowadzania wizyty (rys.~\ref{fig:agenda:conduct}). Do
tego widoku mają dostęp także przypisani do danej agendy ministranci poprzez
kafelek z listą nadchodzących agend na stronie głównej.

\begin{wrapfigure}[19]{1}{.45\textwidth}
    \centering
    \includegraphics[width=.9\linewidth]{figures/screenshots/SOK_Agenda_conduct.png}
    \caption{Przeprowadzanie wizyty}
    \label{fig:agenda:conduct}
\end{wrapfigure}

W trybie przeprowadzania wizyty dostępna jest lista wszystkich wizyt w
agendzie, dostosowana do widoku mobilnego (większe przyciski, układ pionowy) i
umożliwiająca szybkie poruszanie się po liście wizyt i oznaczanie ich statusów.
Przeprowadzanie wizyty można rozpocząć na kwadrans przed ustawionym w agendzie
(bądź w dniu) czasie rozpoczęcia wizyty. Dopiero wtedy wyświetlany jest
przycisk \textit{Rozpocznij wizytę}. Po jego kliknięciu należy wybrać księdza
(o ile nie został on wcześniej wybrany w agendzie), a następnie oznaczone
zostaje pierwsze mieszkanie.

Kafelek każdej wizyty ma różne oznaczenia kolorystyczne w zależności od jej
statusu:
\begin{itemize}
    \item zwykły — wizyta zaplanowana,
    \item zielone obramowanie — następna wizyta,
    \item żółte obramowanie — trwająca wizyta (ksiądz jest w mieszkaniu),
    \item wyszarzony kafelek — zgłoszenie odwiedzone (wizyta zakończona),
    \item czerwone tło — wizyta nieodbyta (odrzucona),
    \item żółte tło — wizyta wstrzymana (tymczasowy status).
\end{itemize}

Status wizyty wyświetlany jest także zgłaszającemu w jego panelu zgłoszenia.
Dodatkowo po rozpoczęciu wizyty przewidywane godziny są aktualizowane na
bieżąco i wyświetlane zgłaszającemu (jeśli w agendzie jest włączona opcja
pokazywania godzin).

Dolny pasek zawiera przyciski do oznaczania liczby domowników w danym
mieszkaniu oraz przyciski do kontroli całości — od lewej: oznaczenie wizyty
jako odrzuconej, dodanie mieszkania (gdy ktoś niezaplanowany poprosi o kolędę)
oraz oznaczenie wizyty jako zakończonej (odwiedzonej). Wszystkie te przyciski
odnoszą się do wizyty, oznaczonej zieloną ramką (następnej). Po oznaczeniu
wizyty jako zakończonej lub odrzuconej, automatycznie zaznaczana jest kolejna
wizyta.

Przycisk do dodawania mieszkania otwiera okienko dodawania wizyty do agendy, w
którym należy podać numer mieszkania (można dodawać tylko mieszkania z
aktualnie odwiedzanej bramy). Po dodaniu mieszkania zostaje ono od razu
oznaczone jako następne.

Więcej specjalistycznych opcji dostępnych jest w menu rozwijanym w prawym
górnym rogu ekranu (ikona trzech pionowych kropek).

\subsection*{Ustawienia aplikacji}

Aby przejść do ustawień aplikacji, należy w bocznym menu wybrać opcję
\textit{Ustawienia} (oznaczoną przez ikonę koła zębatego). Wyświetli się wtedy
strona z kilkoma zakładkami ustawień. Należy wybrać zakładkę \textit{Ustawienia
    ogólne}, aby wyświetlić listę ustawień ogólnych aplikacji dla tej parafii
(rys.~\ref{fig:settings:general}).

\begin{figure}[h]
    \centering
    \includegraphics[width=.75\linewidth]{figures/screenshots/SOK_Settings_general.png}
    \caption{Ustawienia ogólne aplikacji}
    \label{fig:settings:general}
\end{figure}

W tym miejscu można dowolnie je wszystkie zmieniać, a następnie zapisać zmiany
przyciskiem \textit{Zapisz zmiany} w przyklejonym nagłówku bądź je odrzucić
przyciskiem \textit{Cofnij zmiany} obok.

\subsection*{Zarządzanie adresami}

Po przejściu do ustawień ogólnych i wejściu w zakładkę \textit{Adresy} dostępna
jest lista wszystkich wprowadzonych do systemu bram i ulic oraz możliwość ich
tworzenia, edycji i usuwania (rys.~\ref{fig:manage_addresses:list}). Formularze
zgłoszeniowe (publiczny i wewnętrzny) pozwalają na wybór adresu jedynie spośród
wprowadzonych do systemu adresów, zatem istotne jest, aby przed rozpoczęciem
przyjmowania zgłoszeń utworzyć wszystkie potrzebne bramy w parafii.

\begin{figure}[h]
    \centering
    \begin{subfigure}{.5\textwidth}
        \centering
        \includegraphics[width=.95\linewidth]{figures/screenshots/SOK_Addresses_list.png}
        \caption{Lista adresów}
        \label{fig:manage_addresses:list}
    \end{subfigure}%
    \begin{subfigure}{.5\textwidth}
        \centering
        \includegraphics[width=.95\linewidth]{figures/screenshots/SOK_Addresses_edit.png}
        \caption{Edycja bramy}
        \label{fig:manage_addresses:edit}
    \end{subfigure}
    \caption{Zarządzanie adresami}
    \label{fig:manage_addresses}
\end{figure}

Z racji, że może to być dość czasochłonne zadanie, aplikacja umożliwia
tworzenie serii bram przy danej ulicy na podstawie podanego zakresu numerów
(rys.~\ref{fig:manage_addresses:edit}). Na przykład, aby utworzyć bramy dla
ulicy \textit{Kwiatowej} z numerami od 1 do 20, należy najpierw utworzyć ulicę
\textit{Kwiatową}, uzupełniając formularz tworzenia ulicy dostępny po
kliknięciu przycisku \textit{Nowa ulica}, a następnie rozwinąć kafelek tej
ulicy na liście ulic i kliknąć przycisk \textit{Utwórz budynek}. Spowoduje to
wyświetlenie formularza tworzenia budynku, z którego można przejść do tworzenia
bram dla podanego zakresu numerów, klikając przycisk \textit{Dodaj wiele
    budynków na raz}.

Przy tworzeniu i edycji bram dostępne są także opcje zaawansowane, przede
wszystkim określenie dostępności windy. W obecnej wersji aplikacji opcja ta
determinuje kolejność automatycznego układania mieszkań w danej bramie podczas
planowania wizyt (w bramach z windą są porządkowane malejąco).