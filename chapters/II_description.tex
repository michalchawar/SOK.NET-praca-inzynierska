\chapter{Opis i analiza zagadnienia}
\label{sec:rozdzial_II}

Zagadnienie organizacji wizyty duszpasterskiej w modelu \emph{kolędy na
      zaproszenie} jest dosyć skomplikowane i wymaga szeregu funkcjonujących i ściśle
ze sobą współpracujących narzędzi (przede wszystkim do zbierania zgłoszeń,
układania planów i informowania o nich). Dodatkowo potrzeba również personelu,
który zajmie się zarówno wstępnym zaplanowaniem kolędy i ustaleniem
harmonogramu, jak i późniejszą koordynacją przeprowadzania wizyty według planu.

Brak któregokolwiek z tych elementów znacznie utrudnia, a często nawet
uniemożliwia, organizację kolędy na zaproszenie w sposób efektywny i
zadowalający. Z kolei niewłaściwy dobór tych narzędzi lub nieścisła współpraca
między nimi może prowadzić do licznych problemów organizacyjnych, błędów w
planowaniu i komunikacji z parafianami.

\todo{Jakiś ładny diagramik z potencjalnymi problemami?}

Aby model kolędy na zaproszenie osiągał rzeczywistą przewagę nad modelem
tradycyjnym, wszystkie te elementy muszą być starannie dobrane i nadzorowane. W
wielu parafiach brakuje jednak odpowiednich narzędzi i zasobów. Nie ma też na
rynku dedykowanych rozwiązań informatycznych, które kompleksowo wspierałyby ten
model wizyty duszpasterskiej. Z tego powodu często korzysta się z rozwiązań
prymitywnych, prowizorycznych lub niedostosowanych do specyficznych potrzeb
kolędy na zaproszenie, co z kolei nie tylko prowadzi do licznych trudności
organizacyjnych, ale dodatkowo powoduje frustrację zarówno wśród parafian, jak
i koordynatorów kolędy.

Opisywana aplikacja wychodzi naprzeciw tym wyzwaniom, oferując kompleksowe
narzędzie do zarządzania kolędą, które integruje wszystkie niezbędne funkcje w
jednym miejscu. Dzięki temu parafie mogą skutecznie organizować wizyty
duszpasterskie, minimalizując ryzyko błędów i usprawniając komunikację z
parafianami.

\section{Historyjki użytkownika}

\section{Wymagania ogólne}

Jak wspomniano wcześniej, potrzebne narzędzia do organizacji kolędy muszą być
ze sobą ściśle zintegrowane i współpracujące. Brak takich na rynku, powiązany z
koniecznością przeprowadzenia kolędy, był główną motywacją do stworzenia i
rozwoju opisywanej aplikacji. Już w początkowym etapie projektowania systemu
wyklarowały się najważniejsze wymagania, które musiał spełniać, aby nie tylko
skutecznie pomagać w organizacji kolędy, ale również mieć realną szansę na
szerokie zastosowanie w różnych parafiach.

\subsection{Jedność}

Aplikacja musi integrować wszystkie kluczowe funkcje potrzebne do organizacji
kolędy na zaproszenie w jednym miejscu. Oznacza to, że minimalnie powinna
implementować funkcjonalności:

\begin{itemize}
      \item zbierania zgłoszeń od parafian,
      \item zarządzania zebranymi zgłoszeniami oraz wprowadzania nowych,
      \item tworzenia i edycji harmonogramu.
\end{itemize}

Niemniej jednak biorąc pod uwagę współczesne możliwości technologiczne oraz
dążenie do maksymalnej automatyzacji i uproszczenia procesu organizacji możemy
na podstawie implementacji tych trzech punktów zbudować nowoczesny system,
który dodatkowo będzie oferował wiele innych przydatnych funkcji:

\begin{itemize}
      \item zautomatyzowane i zindywidualizowane powiadamianie mailowe,
      \item portal dla parafian, umożliwiający wgląd w dane swojego zgłoszenia, jego
            status, termin wizyty, a nawet przewidywane godziny,
      \item przeprowadzanie wizyty według planu,
      \item zarządzanie personelem zaangażowanym w kolędę,
      \item raportowanie i statystyki dotyczące przebiegu kolędy.
\end{itemize}

Powyższy zestaw funkcji zebrany w jednym miejscu pozwala na kompleksowe
zarządzanie kolędą oraz zapewnia znaczną przewagę nad rozbitymi,
prowizorycznymi narzędziami, które często są wykorzystywane w parafiach.

\subsection{Intuicyjność}

Aplikacja powinna być prosta i intuicyjna w obsłudze, tak aby nawet osoby
nieposiadające zaawansowanych umiejętności technicznych mogły z niej skutecznie
korzystać. Interfejs użytkownika musi być przejrzysty, zrozumiały i jednolity.

Należy jednak dołożyć wszelkich starań, aby nie upraszczać zbytnio
funkcjonalności aplikacji kosztem jej użyteczności. Wdrożenie zbyt
ograniczonego zestawu funkcji może sprawić, że aplikacja nie będzie w stanie
spełnić wszystkie potrzeby parafii, co z kolei może prowadzić do jej odrzucenia
na rzecz bardziej rozbudowanych, choć mniej zintegrowanych rozwiązań.

\subsection{Poprawność}

Aplikacja musi działać niezawodnie i bezbłędnie, zapewniając poprawne
funkcjonowanie wszystkich swoich funkcji. Wszelkie błędy lub awarie mogą
prowadzić do poważnych problemów organizacyjnych, a nawet do utraty zaufania
parafian. Dlatego tak ważne jest, aby aplikacja była starannie przetestowana i
regularnie aktualizowana, aby zapewnić jej stabilność i niezawodność.

Oprócz samej poprawności implementacji aplikacji należy wziąć pod uwagę, że
dane wprowadzone do systemu muszą być również poprawne i spójne. W tym celu
aplikacja powinna implementować mechanizmy walidacji danych, które zapewnią, że
wprowadzone informacje są zgodne z określonymi standardami i nie zawierają
błędów.

Ostatecznie aplikacja będzie wykorzystywana przez koordynatorów kolędy, którzy
mogą nie mieć zaawansowanych umiejętności technicznych. Z tego powodu należy
przyłożyć szczególną uwagę do upewnienia się, że operacje dostępne z poziomu
interfejsu użytkownika są dobrze przemyślane i nie prowadzą do niezamierzonych
konsekwencji, a także dobrze opisane, aby użytkownicy mogli z nich korzystać
bez ryzyka popełnienia błędów.

\subsection{Reużywalność}

Aplikacja powinna być zaprojektowana w sposób umożliwiający jej łatwe
dostosowanie i ponowne wykorzystanie w różnych parafiach. Oznacza to, że
powinna być elastyczna i konfigurowalna, a także umożliwiać coroczne
przeprowadzanie wizyt kolędowych bez konieczności manualnej rekonfiguracji lub
ponownego wdrażania. Dzięki temu parafie będą mogły korzystać z aplikacji przez
wiele lat, co znacznie zwiększy jej wartość i użyteczność.

Sam fakt wykorzystania w różnych parafiach wymusza również konieczność
uwzględnienia tej możliwości w systemie, aby różne parafie mogły korzystać z
aplikacji jednocześnie, bez konieczności tworzenia odrębnych instancji lub
wersji aplikacji dla każdej z nich.

\section{Wymagania funkcjonalne}

Podsumowując powyższe rozważania, aplikacja powinna spełniać następujące
wymagania funkcjonalne:

\begin{itemize}
      \item umożliwiać parafianom zgłaszanie chęci uczestnictwa w kolędzie na zaproszenie
            poprzez formularz online,
      \item pozwalać koordynatorom kolędy na przeglądanie, edytowanie i zarządzanie
            zebranymi zgłoszeniami,
      \item umożliwiać tworzenie i edycję harmonogramu wizyt duszpasterskich na podstawie
            zebranych zgłoszeń,
      \item automatycznie wysyłać powiadomienia mailowe do parafian o potwierdzeniu
            przyjęcia zgłoszenia,
      \item umożliwiać manualne wysyłanie powiadomień mailowych z informacjami o zmianie
            danych zgłoszenia, zaplanowaniu zgłoszenia, itp.,
      \item oferować portal dla parafian, gdzie mogą oni sprawdzać status swojego
            zgłoszenia, termin wizyty oraz przewidywane godziny,
      \item wspierać koordynację personelu zaangażowanego w kolędę, umożliwiając
            przypisywanie zadań i monitorowanie postępów,
      \item generować raporty i statystyki dotyczące przebiegu kolędy, takie jak liczba
            zgłoszeń, liczba przeprowadzonych wizyt itp.,
      \item umożliwiać konfigurację i dostosowanie aplikacji do specyficznych potrzeb
            różnych parafii,
      \item zapewniać możliwość corocznego przeprowadzania kolędy bez konieczności
            manualnej rekonfiguracji lub ponownego wdrażania aplikacji,
      \item umożliwiać jednoczesne korzystanie z aplikacji przez różne parafie, bez
            konieczności tworzenia odrębnych instancji lub wersji aplikacji dla każdej z
            nich.
\end{itemize}

\section{Wymagania niefunkcjonalne }

Oprócz wymagań funkcjonalnych, aplikacja powinna również spełniać następujące
wymagania niefunkcjonalne:

\begin{itemize}
      \item być dostępna online, aby parafie i parafianie mogli z niej korzystać z
            dowolnego miejsca i o dowolnym czasie,
      \item być skalowalna, aby mogła obsługiwać rosnącą liczbę użytkowników i zgłoszeń bez
            utraty wydajności,
      \item być bezpieczna, aby chronić dane osobowe parafian i zapewnić poufność
            informacji,
      \item być zgodna z obowiązującymi przepisami dotyczącymi ochrony danych osobowych,
            takimi jak RODO,
      \item być łatwa w utrzymaniu i aktualizacji, aby zapewnić jej długotrwałe
            funkcjonowanie i możliwość dostosowywania do zmieniających się potrzeb parafii.
\end{itemize}