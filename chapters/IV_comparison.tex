\chapter{Porównanie z innymi implementacjami}

Obecnie rynek aplikacji do organizacji wizyt duszpasterskich nie jest wcale
rozwinięty. Parafie, które decydują się na przeprowadzanie wizyty
duszpasterskiej w modelu kolędy na zaproszenie, najczęściej wybierają
rozwiązania albo prowizoryczne, albo prywatne, tworzone i nadzorowane przez
parafian na własną rękę. Nie istnieją jeszcze żadne aplikacje komercyjne
dedykowane do tego celu.

\section{System manualny}

Najmniej technicznym rozwiązaniem jest podejście manualne, polegające na
prowadzeniu listy wizyt w formie papierowej. Często dopuszcza również
zgłoszenia internetowe (np. poprzez formularz Google), jednak ostateczna lista
wizyt jest tworzona ręcznie przez duszpasterza lub inną osobę odpowiedzialną za
organizację kolędy na papierze, czy to poprzez układanie karteczek z nazwiskami
na stole (w dużych parafiach po całych pomieszczeniach), czy też poprzez
wypisywanie listy na tablicy. Takie podejście jest jednak bardzo niewygodne i
podatne na błędy. W przypadku dużej liczby zgłoszeń łatwo o przeoczenie lub
podwójne zapisanie wizyty. Ponadto, w przypadku konieczności zmiany terminu
wizyty, cała lista musi zostać zmodyfikowana ręcznie, co prowadzi do
dodatkowych komplikacji. Ostatecznie oprócz czasu i wysiłku zajmuje to dużo
przestrzeni fizycznej, a lista papierowa może zostać łatwo zgubiona lub
uszkodzona.

\section{System półautomatyczny}

Niektóre parafie decydują się na wykorzystanie arkuszy kalkulacyjnych (np.
Microsoft Excel lub Google Sheets) do zarządzania listą wizyt duszpasterskich.
W tym podejściu lista wizyt jest tworzona i modyfikowana cyfrowo, co pozwala na
łatwiejsze wprowadzanie zmian i zmniejsza ryzyko błędów związanych z ręcznym
zapisywaniem. Arkusze kalkulacyjne oferują funkcje sortowania i filtrowania
danych, co ułatwia organizację wizyt według różnych kryteriów, takich jak data
czy ulica. Jednakże, mimo że podejście to jest bardziej zaawansowane niż
prowadzenie listy na papierze, nadal wymaga ręcznego wprowadzania danych i
aktualizacji listy. Ponadto, arkusze kalkulacyjne nie oferują dedykowanych
funkcji do zarządzania wizytami duszpasterskimi, co może prowadzić do
nieefektywności i komplikacji w organizacji kolędy.

\section{\textit{Adventus}}

W bieżącym roku (2025) w parafii w Ujanowicach (archidiecezja krakowska)
została wdrożona aplikacja webowa \textit{Adventus} (\cite{adventusapp2026}),
stworzona przez zespół młodzieżowych programistów z tej parafii. Aplikacja ta
umożliwia parafianom śledzenie zaplanowanej trasy kolędowej na interaktywnej
mapie. Jest ona publicznie dostępna, jednak w celu sprawdzenia szczegółów
wizyty (takich jak data, wyznaczony duszpasterz itp.) wymagana jest
autentykacja parafianina poprzez podanie adresu i nazwiska rodziny.

\begin{figure}[ht]
    \centering
    \includegraphics[width=0.8\textwidth]{figures/screenshots/EXT_Adventus.png}
    \caption{Mapa trasy kolędowej w aplikacji \textit{Adventus}}
\end{figure}

Ze względu na zamknięty charakter aplikacji oraz brak dostępu do jej kodu
źródłowego, nie jest wiadome, czy posiada ona funkcje takie jak planowanie
wizyt w sposób w pełni cyfrowy, czy też wymaga ręcznego przenoszenia danych do
innego systemu przez administratora. Nie jest też jasne, czy aplikacja oferuje
takie funkcjonalności jak:

\begin{itemize}
    \item Przewidywanie godzin wizyty,
    \item Powiadamianie mailowe,
    \item Zarządzanie harmonogramami.
\end{itemize}

W systemie funkcjonuje jednak oznaczanie statusów wizyt na bieżąco przez
ministrantów podczas trwania kolędy, co jest źródłem aktualnych informacji dla
parafian (\cite{pytanienasniadanie2026}).

Główną różnicą między systemem \textit{Adventus} a opisaną w tej pracy
aplikacją jest to, że \textit{Adventus} skupia się głównie na udostępnianiu
informacji parafianom, podczas gdy aplikacja opisana w tej pracy ma na celu
ułatwienie organizacji i zarządzania wizytami duszpasterskimi od strony
administracyjnej, co w efekcie poprawia również doświadczenie parafian.

Warto też zwrócić uwagę na różnicę w podejściu do autentykacji użytkowników. W
aplikacji \textit{Adventus} parafianie muszą podać swoje dane (adres i
nazwisko) w celu uzyskania dostępu do szczegółów wizyty, co może budzić obawy
dotyczące prywatności i bezpieczeństwa danych, zwłaszcza w małych
miejscowościach, gdzie wszyscy się znają. Dodatkowo publicznie udostępnione są
informacje o trasie kolędowej, co może prowadzić do różnych niepożądanych, a
czasem i niebezpiecznych sytuacji (szczególnie w obszarach o większym
zagęszczeniu ludności).